% -*-coding: utf-8 -*-
%%***********************************************
%% Plantilla para TFG.
%% Escuela Técnica Superior de Ingenieros Informáticos. UPM.
%%***********************************************
%% Preámbulo del documento.
%%***********************************************
\documentclass[a4paper,11pt]{book}
\usepackage[utf8]{inputenc}
\usepackage[T1]{fontenc}
\usepackage[english,spanish,es-lcroman]{babel}
\usepackage{bookman}
\decimalpoint
\usepackage{graphicx}
\usepackage[font=scriptsize,labelfont=bf, center]{caption}
\usepackage{amsfonts,amsgen,amsmath,amssymb}
\usepackage[top=3cm, bottom=3cm, right=2.54cm, left=2.54cm]{geometry}
\usepackage{afterpage}
\usepackage{colortbl,longtable}
\usepackage[pdfborder={0 0 0}]{hyperref} 
\usepackage{pdfpages}
\usepackage{url}
\usepackage[stable]{footmisc}
\usepackage{parskip} % para separar párrafos con espacio.
\usepackage{hyperref}
%\hypersetup{
%    colorlinks=true,
%    linkcolor=blue,
 %   filecolor=magenta,      
%    urlcolor=blue,
%}
\usepackage{xcolor}
\newcommand{\link}[1]{{\color{blue}\href{#1}{#1}}}
\usepackage{color}

%%---------------------------------------------------------------------------------------
% Custom colors
\usepackage{color}
\definecolor{deepblue}{rgb}{0,0,0.5}
\definecolor{deepred}{rgb}{0.6,0,0}
\definecolor{deepgreen}{rgb}{0,0.5,0}
%%---------------------------------------------------------------------------------------
\usepackage{listings}
\lstset{
    language=bash, %% Troque para PHP, C, Java, etc... bash é o padrão
    basicstyle=\ttfamily\small,
    numberstyle=\footnotesize,
    numbers=left,
    backgroundcolor=\color{gray!10},
    frame=single,
    tabsize=2,
    rulecolor=\color{black!30},
    title=\lstname,
    escapeinside={\%*}{*)},
    breaklines=true,
    breakatwhitespace=true,
    framextopmargin=2pt,
    framexbottommargin=2pt,
    inputencoding=latin2,
    extendedchars=true,
    literate={á}{{\'a}}1 {ã}{{\~a}}1 {é}{{\'e}}1,
}
%%---------------------------------------------------------------------------------------
\usepackage{lmodern}
\usepackage{parskip}

\usepackage{amsmath}
\usepackage{amssymb}

\usepackage{tcolorbox}
\tcbuselibrary{listingsutf8}
\newtcolorbox{mybox}[1]{colback=black!5!white,colframe=black!75!black,
	fonttitle=\bfseries ,title=#1 
	}
%%---------------------------------------------------------------------------------------
\usepackage{fancyhdr}
\pagestyle{fancy}
\fancyhf{}
\fancyhead[LO]{\leftmark}
\fancyhead[RE]{\rightmark}
\setlength{\headheight}{1.5\headheight}
\cfoot{\thepage}

\addto\captionsspanish{ \renewcommand{\contentsname}
  {Tabla de contenidos} }
\setcounter{tocdepth}{4}
\setcounter{secnumdepth}{4}

\renewcommand{\chaptermark}[1]{\markboth{\textbf{#1}}{}}
\renewcommand{\sectionmark}[1]{\markright{\textbf{\thesection. #1}}}
\newcommand{\HRule}{\rule{\linewidth}{0.5mm}}
\newcommand{\bigrule}{\titlerule[0.5mm]}

\usepackage{appendix}
\renewcommand{\appendixname}{Anexos}
\renewcommand{\appendixtocname}{Anexos}
%\renewcommand{\appendixpagename}{Anexos}
%%-----------------------------------------------
%% Páginas en blanco sin cabecera:
%%-----------------------------------------------
\usepackage{dcolumn}
\newcolumntype{.}{D{.}{\esperiod}{-1}}
\makeatletter
\addto\shorthandsspanish{\let\esperiod\es@period@code}

\def\clearpage{
  \ifvmode
    \ifnum \@dbltopnum =\m@ne
      \ifdim \pagetotal <\topskip
        \hbox{}
      \fi
    \fi
  \fi
  \newpage
  \thispagestyle{empty}
  \write\m@ne{}
  \vbox{}
  \penalty -\@Mi
}
\makeatother
%%-----------------------------------------------
%% Estilos código de lenguajes: Consola, C, C++ y Python
%%-----------------------------------------------
\usepackage{color}

\definecolor{gray97}{gray}{.97}
\definecolor{gray75}{gray}{.75}
\definecolor{gray45}{gray}{.45}

\usepackage{listings}
\lstset{ frame=Ltb,
     framerule=0pt,
     aboveskip=0.5cm,
     framextopmargin=3pt,
     framexbottommargin=3pt,
     framexleftmargin=0.4cm,
     framesep=0pt,
     rulesep=.4pt,
     backgroundcolor=\color{gray97},
     rulesepcolor=\color{black},
     %
     stringstyle=\ttfamily,
     showstringspaces = false,
     basicstyle=\scriptsize\ttfamily,
     commentstyle=\color{gray45},
    % keywordstyle=\bfseries,
     %
     numbers=left,
     numbersep=6pt,
     numberstyle=\tiny,
     numberfirstline = false,
     breaklines=true,
     %
     emphstyle=\ttb\color{deepred},    % Custom highlighting style
	 stringstyle=\color{deepgreen},
	 keywordstyle=\color{deepblue},
   }
\lstnewenvironment{listing}[1][]
   {\lstset{#1}\pagebreak[0]}{\pagebreak[0]}

\lstdefinestyle{consola}
   {basicstyle=\scriptsize\bf\ttfamily,
    backgroundcolor=\color{gray75},    
      belowcaptionskip=1\baselineskip,
%  frame=L,
  xleftmargin=\parindent,
  basicstyle=\tiny, %\ttfamily,
    }

\lstdefinestyle{CodigoC}
   {basicstyle=\scriptsize,
	frame=single,
	language=C,
	numbers=left
   }
   
\lstdefinestyle{CodigoC++}
   {basicstyle=\small,
	frame=single,
	backgroundcolor=\color{gray75},
	language=C++,
	numbers=left
   }

\lstdefinestyle{Python}
   {language=Python,    
   }
\makeatother   

%%--------------------------------------------------------------------------------------------
\lstset{language = Python}
%,caption={Descriptive Caption Text},label=DescriptiveLabel}
%\lstdefinestyle{Python2}{
 % belowcaptionskip=1\baselineskip,
%  breaklines=true,
%  frame=L,
%  xleftmargin=\parindent,
%  language=C,
%  showstringspaces=false,
%  basicstyle=\footnotesize\ttfamily,
%  keywordstyle=\bfseries\color{green!40!black},
%  commentstyle=\itshape\color{purple!40!black},
%  identifierstyle=\color{blue},
%  stringstyle=\color{orange},
%}

\lstdefinestyle{Python1}{
  belowcaptionskip=1\baselineskip,
  frame=L,
  xleftmargin=\parindent,
  %language=[x86masm]Assembler,
  language=Python,
  basicstyle=\tiny\ttfamily,
  commentstyle=\itshape\color{purple!40!black},
   %keywordstyle=\bfseries\color{green!40!black},
  %identifierstyle=\color{blue},
  %stringstyle=\color{orange}  ,
     emphstyle=\ttb\color{deepred},    % Custom highlighting style
	 stringstyle=\color{deepgreen},
	 keywordstyle=\color{deepblue},
}
%%--------------------------------------------Para las tildes-------------------------------------------
\lstset{literate=
  {á}{{\'a}}1 {é}{{\'e}}1 {í}{{\'i}}1 {ó}{{\'o}}1 {ú}{{\'u}}1
  {Á}{{\'A}}1 {É}{{\'E}}1 {Í}{{\'I}}1 {Ó}{{\'O}}1 {Ú}{{\'U}}1
  {à}{{\`a}}1 {è}{{\`e}}1 {ì}{{\`i}}1 {ò}{{\`o}}1 {ù}{{\`u}}1
  {À}{{\`A}}1 {È}{{\'E}}1 {Ì}{{\`I}}1 {Ò}{{\`O}}1 {Ù}{{\`U}}1
  {ä}{{\"a}}1 {ë}{{\"e}}1 {ï}{{\"i}}1 {ö}{{\"o}}1 {ü}{{\"u}}1
  {Ä}{{\"A}}1 {Ë}{{\"E}}1 {Ï}{{\"I}}1 {Ö}{{\"O}}1 {Ü}{{\"U}}1
  {â}{{\^a}}1 {ê}{{\^e}}1 {î}{{\^i}}1 {ô}{{\^o}}1 {û}{{\^u}}1
  {Â}{{\^A}}1 {Ê}{{\^E}}1 {Î}{{\^I}}1 {Ô}{{\^O}}1 {Û}{{\^U}}1
  {Ã}{{\~A}}1 {ã}{{\~a}}1 {Õ}{{\~O}}1 {õ}{{\~o}}1
  {œ}{{\oe}}1 {Œ}{{\OE}}1 {æ}{{\ae}}1 {Æ}{{\AE}}1 {ß}{{\ss}}1
  {ű}{{\H{u}}}1 {Ű}{{\H{U}}}1 {ő}{{\H{o}}}1 {Ő}{{\H{O}}}1
  {ç}{{\c c}}1 {Ç}{{\c C}}1 {ø}{{\o}}1 {å}{{\r a}}1 {Å}{{\r A}}1
  {€}{{\euro}}1 {£}{{\pounds}}1 {«}{{\guillemotleft}}1
  {»}{{\guillemotright}}1 {ñ}{{\~n}}1 {Ñ}{{\~N}}1 {¿}{{?`}}1 
  {Ý}{{\'Y}}1 {º}{{$^{\circ}$}}1 ,
  numbers=left, 
  numberstyle=\tiny,
	numberfirstline=false, 
	numbersep=3pt %al reves, cuanto 8 se queda en medio de la linea, 5 queda justo
}
%Another possibility is to replace \usepackage{listings} (in the preamble) with \usepackage{listingsutf8}, but this will only work for \lstinputlisting{...}.
\usepackage{listings}







