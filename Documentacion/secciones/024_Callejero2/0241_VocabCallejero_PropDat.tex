\subsection{Propiedades de datos}
%----------------------------------------------------------------------------------------------------------------------------------------------------------------------------------------------------------------------------------------------
%----------------------------------------------------------------------------------------------------------------------------------------------------------------------------------------------------------------------------------------------
%----------------------------------------------------------------------------------------------------------------------------------------------------------------------------------------------------------------------------------------------
%----------------------------------------------------------------------------------------------------------------------------------------------------------------------------------------------------------------------------------------------




\begin{mybox}{esCalleTranquila}
\begin{flushleft}
\underline{\textbf{IRI:}}
\url{http://vocab.linkeddata.es/datosabiertos/def/urbanismo-infraestructuras/callejero#esCalleTranquila}
\newline

Propiedad que indica si una vía es calle tranquila o no para bicicletas. Vías con poco tráfico, mucha visibilidad o con mucho porcentaje de accidentes pueden ser algunos de los criterios seguidos para esta valoración.
\newline


\underline{\textbf{Definida por:}}\newline
\url{http://vocab.linkeddata.es/datosabiertos/def/urbanismo-infraestructuras/callejero}
\newline

\underline{\textbf{Dominio:}}
	Via
\newline

\underline{\textbf{Rango:}}
		xsd:boolean

\end{flushleft}
\end{mybox}
%----------------------------------------------------------------------------------------------------------------------------------------------------------------------------------------------------------------------------------------------





\begin{mybox}{longitud}
\begin{flushleft}
\underline{\textbf{IRI:}}
\url{http://vocab.linkeddata.es/datosabiertos/def/urbanismo-infraestructuras/callejero#longitud}
\newline

Longitud de la calle o tramo de la calle descrito. Su unidad de medida es el metro aunque en muchos casos puede venir representado como Shape$\_$leng.
\newline

\underline{\textbf{Definida por:}}\newline
\url{http://vocab.linkeddata.es/datosabiertos/def/urbanismo-infraestructuras/callejero}
\newline

\underline{\textbf{Dominio:}}
		Via
\newline

\underline{\textbf{Rango:}}
		xsd:double

\end{flushleft}
\end{mybox}
%----------------------------------------------------------------------------------------------------------------------------------------------------------------------------------------------------------------------------------------------







\begin{mybox}{dobleSentido}
\begin{flushleft}
\underline{\textbf{IRI:}}
\url{http://vocab.linkeddata.es/datosabiertos/def/urbanismo-infraestructuras/callejero#dobleSentido}
\newline

Propiedad que determina si una vía es de sentido único o doble.
%Puede tomar los siguientes valores definidos en datos.ign.es \cite{datosIgn_calzada}:
%\newline 1: Calle de doble sentido.
%\newline 0: Calle de sentido único.
%\newline { A la espera de conocer su significado oficial, basado en datos aproximados }
%\newline \url{http://vocab.ciudadesabiertas.es/kos/urbanismo-infraestructuras/calle/doble-sentido/SENTIDO-UNICO}
%\newline \url{http://vocab.ciudadesabiertas.es/kos/urbanismo-infraestructuras/calle/doble-sentido/DOBLE-SENTIDO}
\newline

\underline{\textbf{Definida por:}}\newline
\url{http://vocab.linkeddata.es/datosabiertos/def/urbanismo-infraestructuras/callejero}
\newline

\underline{\textbf{Dominio:}}
		Via
\newline

\underline{\textbf{Rango:}}
 xsd: boolean
 \newline

\end{flushleft}
\end{mybox}
%----------------------------------------------------------------------------------------------------------------------------------------------------------------------------------------------------------------------------------------------












