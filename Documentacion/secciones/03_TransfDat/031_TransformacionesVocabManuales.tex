
\section{Modificaciones manuales en Datasets}

Como ya se ha descrito anteriormente los ficheros se deben visualizar en codificación ISO-8859-1. Esto no resuelve todos los errores debido a que ciertas palabras siguen mostrándose de manera incorrecta. Para resolver este inconveniente se han realizado modificaciones automáticas (descritas en los siguientes capítulos, para los elementos más comunes) y otras manuales (en elementos no repetitivos y fáciles de cambiar). Estos errores se encuentran mayoritariamente en el dataset de Ciclocarriles, para el cual no fue posible encontrar la codificación adecuada y no contenía demasiados registros (unos 150).
\newline

%------------------------------- CICLOCARRILES --------------------------------------------------
\begin{itemize}

\item Cambios realizados de forma manual al conjunto de datos de Ciclocarriles:
\begin{tiny}
\newline - ln 52: M$/$ndez ?lvaro--> Méndez Álvaro
\newline - ln 64-65: Men$/$ndez Pelayo --> Menéndez Pelayo
\newline - ln 72-73: Ortega y Gasset --> Jose Ortega y Gasset (Igual al nombre del Callejero de Madrid)
\newline - ln 103: Donoso Cort$/$s --> Donoso Cortés
\newline - ln 112: Gral Moscardæ --> Gral Moscardó
\newline - ln 114: Camilo Jos$/$Cela - Azcona --> Camilo José Cela (También eliminado Azcona ya que no esta previsto la existencia de cruces)
\newline - ln 117: Gral Yag$>$e --> Gral Yagüe
\newline - ln 125: MARQUÉS DE VIANA - SOR ANGELA DE LA CRUZ --> Dividido en 2 registros con características similares.
\newline
\end{tiny}

%------------------------------- CALLES TRANQUILAS ---------------------------------------------

    \item Cambios realizados de forma manual al dataset de CallesTranquilas:
\begin{tiny}
\newline - ln 1292-1292: Calle de la Cooperativa ElÚctrica --> Calle de la Cooperativa Eléctrica
\newline - ln 1822-1823: Doctor MartÝn ArÚvalo --> Doctor Martín Arévalo
\newline - ln 1919-1921: Errores en el formato del csv o de codificación. Mismo registro en varias lineas.
\newline     - ln 1673: AVENIDA ALBUFERA CON FELIPE ÁLVAREZ --> AVENIDA ALBUFERA
% Esta la dejo asi porque tendria que eliminarla porque es de cruces
\newline - ln 2040: ENLACE CALLE AMERICIO CON MADRID RÍO --> CALLE AMERICIO
- ln 1716: PARQUE LINEAL DE PALOMERAS CON GONZÁLEZ DÁVILA --> Eliminada (peatonal)
-ln 1846: MARMOLINA CON AVENIDA COMUNIDADES --> Eliminada
\end{tiny}


Para la realización de estos cambios se ha observado el mapa proporcionado por el ayuntamiento \cite{mapa_callejero_klm} y se ha determinado la mejor forma de representar los datos. En los casos que han sido eliminadas o que formaban partes de cruces y se ha mantenido únicamente una calle, se ha tomado en consideración el mapa proporcionado en el url anterior y se ha considerado que era la mejor manera de representar esos datos o que no eran relevantes.



%------------------------------- ACCIDENTES ---------------------------------------------------
%-------------------------------------------------------------------------------------------------




%------------------------------- CALLEJERO ---------------------------------------------------


    \item Cambios realizados de forma manual al dataset del Callejero:
\begin{tiny}
\newline 201600;CALLE;DEL;COMANDANTE ZORITA;AVIADOR ZORITA;6;1;59;2;50 --> Igual que el registro ``Aviador Zorita``
\newline 334200;CALLE;DE;GENERAL YAGUE;GENERAL YAGÜE;6;1;57;4;76 --> Igual que el registro ``San German``. Cambio de nombre de la vía posterior a la realización de varios dataserts \url{https://es.wikipedia.org/wiki/Calle_de_San_Germán}.
\newline 331600;CALLE;DE;GENERAL MOSCARDO;GENERAL MOSCARDÓ;6;1;39;2;34 --> Igual que el registro ``Edgar Neville``. Cambio de nombre de la via posterior a la realizacion de varios dataserts \url{https://www.elmundo.es/madrid/2017/05/31/592dbf00e2704ed5058b4688.html}.
\newline 765800;RONDA;DE;RONDA VALENCIA;RONDA VALENCIA;1;;;2;18 --> Se considera nombre completo ``Ronda de Valencia``, y no separado como muestra inicialmente
\end{tiny}

Estos cambios se realizan directamente en el dataset del callejero ya que pueden ser aplicables a todos los datasets. Elementos que se consideran básicos en casos concretos, calles nuevas o nuevos nombres (como es el caso de algunos referidos a personajes militares o políticos) cambiados en los últimos años, deben añadirse por si no han sido actualizados en algunos casos, conservando ambos.\newline
Para ello se ha seguido la lista proporcionada por El Pais \cite{calles_cambioNombre_elPais} y se han añadido tanto los cambios ya realizados, como los aprobados aun no actualizados en el dataset, para que estén ambos nombres.

Los elementos añadidos se encuentran en el Anexo.













Para su adición se han obtenido las propiedades correspondientes a su nombre antiguo de forma que correspondan a la misma vía.









\end{itemize}






