\chapter{Definición de Vocabularios}

En este proyecto se han definido 2 vocabularios y se ha realizado una propuesta de modificación para el Callejero \cite{ciudadesbiertas_callejero}. Para ello se han reutilizado elementos de otras ontologías y se han generado nuevas clases y propiedades que han permitido representar de forma clara y detallada los nuevos elementos propuestos.


Se han tomado en cuenta los datasets proporcionados por el Ayuntamiento de Madrid \cite{datosabiertos_ayuntmadrid} para ello y se han definido acorde con los datos que se estaban proporcionando en los mismos. De esta forma se ha pretendido seguir la linea marcada por la institución para la publicación de los mismos con la intención de interferir lo mínimo en el proceso de creación ya definido.


Como norma general los vocabularios han de seguir un proceso de coordinación entre instituciones y deben realizarse mediante consenso entre desarrolladores y entidades proveedoras de los mismos. Para este proyecto no ha sido posible dicho estudio exhaustivo sobre cada una de las propiedades y clases, es por ello que se ha intentado asemejar lo máximo a los valores proporcionados por el Ayuntamiento de Madrid y se han realizado las propuestas a la plataforma ciudadesabiertas \cite{ciudadesabiertas_catalogoVocabs}.
