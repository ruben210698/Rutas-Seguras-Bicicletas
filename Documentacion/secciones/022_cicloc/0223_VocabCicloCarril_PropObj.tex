\subsection{Propiedades de objeto}
%----------------------------------------------------------------------------------------------------------------------------------------------------------------------------------------------------------------------------------------------
%----------------------------------------------------------------------------------------------------------------------------------------------------------------------------------------------------------------------------------------------
%----------------------------------------------------------------------------------------------------------------------------------------------------------------------------------------------------------------------------------------------
%----------------------------------------------------------------------------------------------------------------------------------------------------------------------------------------------------------------------------------------------


\begin{mybox}{tieneCalle}
\begin{flushleft}
\underline{\textbf{IRI:}}
\url{http://vocab.linkeddata.es/datosabiertos/def/urbanismo-infraestructuras/ciclo-carril#tieneCalle}
\newline

Calle asociada a un CicloCarril.
\newline

\underline{\textbf{Definida por:}}
\url{http://vocab.linkeddata.es/datosabiertos/def/urbanismo-infraestructuras/ciclo-carril}
\newline

\underline{\textbf{Dominio:}}
		CicloCarril
\newline

\underline{\textbf{Rango:}}
		Calle
\newline


\end{flushleft}
\end{mybox}
%----------------------------------------------------------------------------------------------------------------------------------------------------------------------------------------------------------------------------------------------



\begin{mybox}{tipoUso}
\begin{flushleft}
\underline{\textbf{IRI:}}
\url{http://vocab.linkeddata.es/datosabiertos/def/urbanismo-infraestructuras/ciclo-carril#tipoUso}
\newline

Identificador del tipo de uso que puede tener la calle. Se han definido 2 clases para ello:
\newline -	\url{http://vocab.ciudadesabiertas.es/kos/urbanismo-infraestructuras/calle/tipo-uso/CICLOCALLE}
\newline -	 \url{http://vocab.ciudadesabiertas.es/kos/urbanismo-infraestructuras/calle/tipo-uso/PEATONAL}
\newline


%\underline{\textbf{Definida por:}}
%\url{http://vocab.ciudadesabiertas.es/def/calle-bici}
%\newline

\underline{\textbf{Dominio:}}
\newline CicloCarril

\underline{\textbf{Rango:}}
		concept


\end{flushleft}
\end{mybox}
%----------------------------------------------------------------------------------------------------------------------------------------------------------------------------------------------------------------------------------------------






\begin{mybox}{tieneIdVia}
\begin{flushleft}
\underline{\textbf{IRI:}}
\url{http://vocab.linkeddata.es/datosabiertos/def/urbanismo-infraestructuras/ciclo-carril#tieneIdVia}
\newline

Identificador de calle asociado.
\newline

\underline{\textbf{Definida por:}}
\url{http://vocab.linkeddata.es/datosabiertos/def/urbanismo-infraestructuras/ciclo-carril}
\newline

\underline{\textbf{Dominio:}}
		Calle
\newline

\underline{\textbf{Rango:}}
		Via
\newline


\end{flushleft}
\end{mybox}
%----------------------------------------------------------------------------------------------------------------------------------------------------------------------------------------------------------------------------------------------





\begin{mybox}{municipio}
\begin{flushleft}
\underline{\textbf{IRI:}}
\url{http://vocab.linkeddata.es/datosabiertos/def/sector-publico/territorio#municipio}
\newline

Municipio al que pertenece un fenómeno geográfico o una entidad administrativa.
\newline

\underline{\textbf{Definida por:}}
\url{http://vocab.linkeddata.es/datosabiertos/def/sector-publico/territorio}
\newline

\underline{\textbf{Dominio:}}
		CicloCarril
\newline

\underline{\textbf{Rango:}}
		Municipio

\end{flushleft}
\end{mybox}



%----------------------------------------------------------------------------------------------------------------------------------------------------------------------------------------------------------------------------------------------















\begin{mybox}{tieneCarrilExclusBici}
\begin{flushleft}
\underline{\textbf{IRI:}}
\url{http://vocab.linkeddata.es/datosabiertos/def/urbanismo-infraestructuras/ciclo-carril#tieneCarrilExclusBici}
\newline

Indicador de si las bicicletas disponen o no de un carril propio para su circulación.Se han definido 2 clases para ello:
\newline -	\url{http://vocab.ciudadesabiertas.es/recurso/calle-bici/ciclo-carril/carrilExlus/EXCLUSIVO-BICI}
\newline -	 \url{http://vocab.ciudadesabiertas.es/recurso/calle-bici/ciclo-carril/carrilExlus/NO-EXLUSIVO-BICI}
\newline


\underline{\textbf{Definida por:}}
\url{http://vocab.linkeddata.es/datosabiertos/def/urbanismo-infraestructuras/ciclo-carril}
\newline

\underline{\textbf{Dominio:}}
	CicloCarril
\newline

\underline{\textbf{Rango:}}
	concept
\newline

\end{flushleft}
\end{mybox}
%----------------------------------------------------------------------------------------------------------------------------------------------------------------------------------------------------------------------------------------------







