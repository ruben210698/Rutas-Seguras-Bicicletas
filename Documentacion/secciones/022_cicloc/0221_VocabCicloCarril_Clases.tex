\subsection{Clases}



\begin{mybox}{CicloCarril}
\begin{flushleft}
\underline{\textbf{IRI:}}
\url{http://vocab.ciudadesabiertas.es/def/calle-bici/ciclo-carril#CicloCarril}
\newline

Via con uno o más carriles destinados al tránsito de ciclistas. No necesariamente exclusivos para el tránsito de bicicletas, pero si con señalización y limitaciones adaptadas para ello.
\newline

\underline{\textbf{Definida por:}}
\url{http://vocab.ciudadesabiertas.es/def/calle-bici/ciclo-carril}
\newline

\underline{\textbf{Tiene subclase:}}
\newline Calle,\hspace{2em} municipio,\hspace{2em} tipoUso
\newline 

\underline{\textbf{Tiene Superclase:}}
\newline distMaxExclusBici

\end{flushleft}
\end{mybox}

%----------------------------------------------------------------------------------------------------------------------------------------------------------------------------------------------------------------------------------------------




\begin{mybox}{Calle}
\begin{flushleft}
\underline{\textbf{IRI:}}
\url{http://vocab.ciudadesabiertas.es/def/calle-bici/ciclo-carril#Calle}
\newline

Representación de una via de una ciudad.
\newline

\underline{\textbf{Definida por:}}
\url{http://vocab.ciudadesabiertas.es/def/calle-bici/ciclo-carril}
\newline

\underline{\textbf{Tiene subclase:}}
\newline carrilExclusBici,\hspace{2em} Via
\newline 

\underline{\textbf{Tiene Superclases:}}
\newline CicloCarril,\hspace{2em} longitud,\hspace{2em} nombre oficial

\end{flushleft}
\end{mybox}

%----------------------------------------------------------------------------------------------------------------------------------------------------------------------------------------------------------------------------------------------





\begin{mybox}{Via}
\begin{flushleft}
\underline{\textbf{IRI:}}
\url{http://vocab.linkeddata.es/datosabiertos/def/urbanismo-infraestructuras/callejero#Via}
\newline

Se ha reutilizado la definición de Municipio proporcionada por vocab.linkeddata.es \cite{datoabiertos_idVia}

Vía de comunicación construida para la circulación. En su definición según el modelo de direcciones de la Administración General del Estado, Incluye calles, carreteras de todo tipo, caminos, vías de agua, pantalanes, etc. Asimismo, incluye la pseudovía., es decir todo aquello que complementa o sustituye a la vía. En nuestro caso, este término se utiliza para hacer referencia a las vías urbanas.
Representación numérica de la misma.
\newline

\underline{\textbf{Definida por:}}
\url{http://vocab.linkeddata.es/datosabiertos/def/urbanismo-infraestructuras/callejero}
\newline

\underline{\textbf{Tiene Superclases:}}
\newline Calle





\end{flushleft}
\end{mybox}
%----------------------------------------------------------------------------------------------------------------------------------------------------------------------------------------------------------------------------------------------


\begin{mybox}{Municipio}
\begin{flushleft}
\underline{\textbf{IRI:}}
\url{http://vocab.linkeddata.es/datosabiertos/def/sector-publico/territorio#Municipio}
\newline

Se ha reutilizado la definición de Municipio proporcionada por vocab.linkeddata.es \cite{datoabiertos_municipio}
Un Municipio es el ente local definido en el artículo 140 de la Constitución española y la entidad básica de la organización territorial del Estado según el artículo 1 de la Ley 7/1985, de 2 de abril, Reguladora de las Bases del Régimen Local. Tiene personalidad jurídica y plena capacidad para el cumplimiento de sus fines. La delimitación territorial de Municipio está recogida del REgistro Central de Cartografía del IGN. Su nombre, que se especifica con la propiedad dct:title, es el proporcionado por el Registro de Entidades Locales del Ministerio de Política Territorial, en \url{http://www.ine.es/nomen2/index.do}
\newline


\underline{\textbf{Definida por:}}
\newline \url{http://purl.org/derecho/vocabulario}
\newline \url{http://vocab.linkeddata.es/datosabiertos/def/sector-publico/territorio}
\newline \url{http://www.ign.es/ign/resources/acercaDe/tablon/ModeloDireccionesAGE}
\newline

\underline{\textbf{Esta en rango de:}}
\newline municipio

%\underline{\textbf{Tiene Superclases:}}
%\newline CicloCarril



\end{flushleft}
\end{mybox}
%----------------------------------------------------------------------------------------------------------------------------------------------------------------------------------------------------------------------------------------------




%----------------------------------------------------------------------------------------------------------------------------------------------------------------------------------------------------------------------------------------------
%----------------------------------------------------------------------------------------------------------------------------------------------------------------------------------------------------------------------------------------------
%----------------------------------------------------------------------------------------------------------------------------------------------------------------------------------------------------------------------------------------------
%----------------------------------------------------------------------------------------------------------------------------------------------------------------------------------------------------------------------------------------------


