\chapter{Lineas Futuras}

Como se ha mencionado en las conclusiones, los datos enlazados tienen un gran potencial y en este caso se ha hecho uso de 3 fuentes de datos distintas relativas a la seguridad de bicicletas. Esto, aun proporcionando una cantidad de datos amplia, es insuficiente para cálculos fiables y que reflejen correctamente la realidad. Si bien se planteo al inicio de este proyecto, finalmente se descartó por su representación en formato KML y su dificultad para ser tratado, sería de gran interés añadir información sobre tráfico \cite{datos_madrid_trafico}. Con ello se podría determinar las calles más transcurridas, y por tanto, con mayores probabilidades de sufrir accidentes. Se podría añadir también la información sobre el uso de bicicleta pública en Madrid \cite{datosMadrid_rutas_bici}. Con esta información se podría obtener un cálculo mucho más exacto de las probabilidades de sufrir un accidente en una vía. Si bien es cierto que no todas las bicicletas son públicas y por tanto no se tiene su itinerario, éstas si representan un bien porcentaje de las utilizadas en la ciudad. No es lo mismo que ocurra un accidente en una calle por la que circulan 100 bicicletas al día que una por la que transcurren 5. Actualmente la aplicación desarrollada no lo distingue, las valora del mismo modo, esto es claramente erróneo y es por ello que sería de gran utilidad incluirlo en posteriores actualizaciones.

Como se ha comentado en el apartado de las transformaciones, se han intentado hacer los mínimos cambios manuales posibles y se ha priorizado su tratamiento por código, de forma que a todos los datasets se le aplicasen las mismas. Esto, además de aumentar las probabilidades de coincidencia entre elementos, permite aplicarlo a otros datasets similares. En el caso de ciclocarriles no es posible, únicamente si se actualizase el ya existente o si se diseñase para otra ciudad, sin embargo actualmente solo se han añadido los accidentes de bicicletas en 2019. Todas las transformaciones aplicadas a éste podrían hacerse para los conjuntos de datos de años anteriores del mismo tipo \cite{datosMadrid_accidentesDeBicicleta} y de esta forma añadir una gran cantidad de información.

Parte de la información que se ha definido en los vocabularios y que contienen los TTL añadidos a la aplicación no ha sido utilizada. Esto es debido a que el cómputo llevado a cabo ha sido simple, no ha seguido ninguna norma y ha sido utilizado más bien para mostrar la funcionalidad que podría tener. Se podría consultar al usuario la hora a la que va a realizar el trayecto y de esta forma determinar franjas (mañana, tarde y noche) para solo tener en cuanta los accidentes ocurridos en esos márgenes temporales. Con la fecha podría realizarse una comprobación similar para fines de semana y días laborales. Una operación más compleja podría ser la comprobación de la meteorología; comprobar con la API de la AEMET las circunstancias actuales y en función de éstas filtrar los accidentes por los ocurridos con lluvia, despejado, niebla... Estas variables requieren una mayor categorización y un mayor estudio sobre el impacto que podrían tener, y también las que ya se están utilizando, es por ello que sería una buena linea de trabajo para continuar lo desarrollado hasta ahora.

En un aspecto más enfocado a la interfaz gráfica y no tanto al enlazado de datos, se podría mejorar en gran medida la forma en la que se muestra la información y se podría mostrar un mapa con diferentes colores para las calles por las que se transita. En algunos ejemplos las incidencias muestran calles con multitud de accidentes que rebajan la nota 2 puntos. Sería de gran interés recomendar al usuario evitar esa vía, o solicitar una ruta alternativa a Google Maps evitándola (marcando como destino intermedio una calle paralela a esta, por ejemplo).


Los datos enlazados tienen multitud de aplicaciones y cuanta mayor información se disponga, mejor serán los resultados obtenidos. En el portal de datos del Ayuntamiento de Madrid \cite{datosabiertos_ayuntmadrid} hay publicados gran cantidad de datasets, muchos de ellos relacionados con bicicletas, y con un correcto modelado y utilización de los mismos podrían incluirse en éste y otros proyectos.
