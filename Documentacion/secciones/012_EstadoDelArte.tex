\chapter{Estado del Arte}

En este apartado se explicarán brevemente las tecnologías y herramientas utilizadas para el desarrollo de este proyecto. En particular, una aplicación Android basada en datos enlazados.


\section{OpenRefine}
OpenRefine es una herramienta poderosa para trabajar con datos desordenados: limpiarlos, transformarlos de un formato a otro y ampliarlos con servicios web y datos externos.

OpenRefine siempre mantiene sus datos privados en de forma local hasta que se desee compartir o colaborar. Los datos privados nunca salen del dispositivo a menos que se desee. (Funciona ejecutando un pequeño servidor y usa un navegador web para interactuar con él) \cite{pagPrinc_OpenRefine}.

\section{Apache Jena}
Apache Jena es un framework de Java para construir aplicaciones de la Web Semántica. Proporciona un entorno programático para RDF, RDFS y OWL, SPARQL e incluye un motor de inferencia basado en reglas \cite{JenaDescripcion}.

\section{Python}
Python es un lenguaje de propgramación interpretado, orientado a objetos y de alto nivel.
La sintaxis simple y fácil de aprender de Python enfatiza la legibilidad y, por lo tanto, reduce el costo del mantenimiento del programa. Python admite módulos y paquetes, lo que fomenta la modularidad del programa y la reutilización de código. El intérprete de Python y la extensa biblioteca estándar están disponibles en formato fuente o binario, sin cargo para todas las plataformas principales, y se pueden distribuir libremente \cite{pythonDescripcion}.


\section{Java}
Java es un lenguaje de programación y una plataforma informática. Tiene la ventaja de ser ejecutada sobre una máquina virtual que permite su funcionamiento en múltiples plataformas\cite{descripcionJava}. Es gratuito y será utilizado en el proyecto aquí desarrollado para la programación de la interfaz gráfica en Android.


\section{SQLite}
SQLite es una biblioteca que implementa un motor de base de datos SQL transaccional autónomo, sin servidor, sin configuración. El código para SQLite es de dominio público y, por lo tanto, es de uso gratuito para cualquier propósito, comercial o privado\cite{descripcionSQLite}.

\section{SPARQL}
RDF es un formato de datos de gráfico etiquetado y dirigido para representar información en la Web. Esta especificación define la sintaxis y la semántica del lenguaje de consulta SPARQL para RDF. SPARQL se puede utilizar para expresar consultas en RDF. SPARQL contiene capacidades para consultar patrones gráficos requeridos y opcionales. Los resultados de las consultas SPARQL pueden ser conjuntos de resultados o gráficos RDF \cite{sparqlDescription}.

\section{Draw.io}
Es una aplicación gratuita utilizada para diseñar y construir diagramas. En este proyecto se ha utilizado para mostrar de forma detallada los componentes de los vocabularios y representarlos de una forma clara y visual.
