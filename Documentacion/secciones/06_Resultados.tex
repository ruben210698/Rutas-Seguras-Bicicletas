\chapter{Resultados y conclusiones}

Si bien los resultados de este proyecto se pueden comprobar en la interfaz gráfica realizada para ello, han de explicarse más detalladamente para comprender el alcance de los mismos.

El resultado más visible se aprecia en las incidencias que se muestran: se le proporciona el origen y destino y se realiza un listado de ciclocarriles, calles tranquilas y accidentes que han ocurrido en la ruta. De esta información se hace un cómputo para determinar la seguridad del recorrido y se ``aconseja`` o no al usuario si realizarlo en bicicleta o no (proporcionándole una calificación numérica). A grandes rasgos puede parecer una tarea simple, pero el proceso de obtener toda esa informa a partir de dos direcciones ha sido gracias al enlazado de datos y su previa definición.

Para lograr acceder a todas las variables que se utilizan en el cómputo (podrían ser muchas más), se ha accedido a 3 datasets distintos proporcionados por el Ayuntamiento de Madrid, estos a su vez han sido transformados con el callejero (otro dataset), a la API Directions de Google y a una base de datos con las coordenadas y calles del municipio (también proporcionada por el Ayuntamiento). Se han utilizado 6 fuentes de datos distintas para proporcionar un cálculo aproximado sobre la seguridad, con una cantidad de información suficiente para que el cálculo sea fiable.

En este caso no se ha hecho uso de todas las propiedades de las que se disponía, sin embargo podría hacerse y podría enlazarse la aplicación con muchos más datasets relacionados con la seguridad en bicicletas. Esto es posible gracias a que todos ellos están conectados, en este caso por el identificador de la vía, y eso permite que se pueda buscar información en todos ellos y poder dar ese valor añadido que nos proporciona la cantidad de datos que tenemos disponibles sobre cada lugar.

El problema principal del desarrollo de este proyecto, y en general de los datos abiertos, es la falta de entendimiento entre las partes y las diferentes representaciones que se puede dar a una misma información. En concreto en este trabajo los 3 datasets utilizados no contenían el dato ``identificador de vía``, lo cual hace imposible su enlazado con otros conjuntos (excepto en casos puntuales en los que el nombre coincida exactamente). Ante problemas como este se plantea la definición de vocabularios, a modo de estándar para futuros datasets, de forma que la persona o institución que proporciona los datos pueda adaptarse a ellos y proporcionar la información de un modo correcto, permitiendo así su correcta reutilización y fácil tratamiento. No solo se define el modo de proporcionar los datos, también información que se debería añadir y que sería de gran utilidad para aplicaciones en ese sector.

Se ha observado en este proyecto el arduo trabajo que puede ser inferir información necesaria que no contiene el origen de datos. La definición de estos vocabularios permite de algún modo que haya un consenso entre desarrolladores e instituciones para que sea más sencillo su tratamiento y el ciudadano pueda beneficiarse de los resultados obtenidos.



Finalmente, como conclusión personal, el trabajo ha sido muy enriquecedor y provechoso. Se ha construido una aplicación Android que muestra los resultados correctamente y es curioso lo simple que puede parecer en el resultado final. Esta simpleza en parte es por el hecho de haber transformado los datasets para que sean similares a los vocabularios aquí definidos. Gran parte del proyecto ha sido estas transformaciones, en caso de que el ayuntamiento las siguiese para posteriores ocasiones sería relativamente simple utilizarlos y se podrían hacer grandes aplicaciones con ellos, que al fin y al cabo es el objetivo de las instituciones públicas, proporcionar esos datos para que libremente cualquiera pueda utilizarlos.
