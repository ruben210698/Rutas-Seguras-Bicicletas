\chapter{Resumen}

La publicación de datos abierto por parte de instituciones es algo común actualmente y que puede tener un gran potencial. Estos datos públicos pueden ser utilizados para cualquier fin y con multitud de aplicaciones en caso de tratarlos conjuntamente con otros.


Aun siendo de gran utilidad, su práctica no está tan extendida como se podría pensar. Esto es debido a que comunmente no son fácilmente reutilizables y esto limita su tratamiento. Es por ello que se trabaja para definir ``vocabularios`` que permitan a las instituciones que los publican, hacerlo de forma ordenada y similar a otras. El uso de estos estándares permite que enlazar recursos y reutilizar aplicaciones para los conjuntos de datos provenientes de orígenes distintos.


En este contexto se plantea diseñar varias ontologías sobre elementos relacionados con bicicletas y proponerlas para ser incluidas en la plataforma ``ciudadesabiertas``. Estas seguirán los principios de Web Semántica y Linked Data que permita realizar un buen modelado de los datos, para su correcto funcionamiento y para una posible reutilización posterior. Tras esta definición, se transformarán los datasets proporcionados por el Ayuntamiento de Madrid para que sigan estas recomendaciones de modelado. Con estos conjuntos de datos se desarrollará una aplicación que muestre la utilidad de esta información y su enlazado, demostrando el valor añadido de la conexión entre ellos y las posibilidades que podría tener su uso.
\newline

\textbf{Palabras clave:} bicicletas, Web Semántica, Linked Data, ontología

