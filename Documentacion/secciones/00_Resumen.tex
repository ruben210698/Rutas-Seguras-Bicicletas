\chapter{Resumen}

La publicación de datos abierto por parte de instituciones es algo común actualmente y que puede tener un gran potencial. Estos datos públicos pueden ser utilizados para cualquier fin y con multitud de aplicaciones en caso de tratarlos conjuntamente con otros.


Aun siendo de gran utilidad, su práctica no está tan extendida como se podría pensar. Esto es debido a que comunmente no son fácilmente reutilizables y esto limita su tratamiento. Es por ello que se trabaja para definir ``vocabularios`` que permitan a las instituciones que los publican, hacerlo de forma ordenada y similar a otras. El uso de estos estándares permite que enlazar recursos y reutilizar aplicaciones para los conjuntos de datos provenientes de orígenes distintos.


En este contexto se plantea diseñar varias ontologías sobre elementos relacionados con bicicletas y proponerlas para ser incluidas en la plataforma ``ciudadesabiertas``. Estas seguirán los principios de Web Semántica y Linked Data que permita realizar un buen modelado de los datos, para su correcto funcionamiento y para una posible reutilización posterior. Tras esta definición, se transformarán los datasets proporcionados por el Ayuntamiento de Madrid para que sigan estas recomendaciones de modelado. Con estos conjuntos de datos se desarrollará una aplicación que muestre la utilidad de esta información y su enlazado, demostrando el valor añadido de la conexión entre ellos y las posibilidades que podría tener su uso.
\newline

\textbf{Palabras clave:} bicicletas, Web Semántica, Linked Data, ontología







\chapter{Abstract}

Nowadays the publication of open data by institutions is common and have a great potential. These public data can be used for any purpose and with a multitude of applications if they are treated in conjunction with others.


In spite the fact that they are very useful, its practice is not as widespread as one might think. This is because they are generally not easily reusable, which limits their treatment. That is why we are working to define ``vocabularies`` that allow the institutions that publish them to do it in an orderly manner and similar to others. Using these standards allows you to link resources and reuse dataset applications from different sources.


In this context, it is proposed to design various ontologies on bicycle related elements and propose them to be included in the ``opencities`` platform. These will follow the principles of Semantic Web and Linked Data that allow a good modeling of the data, for its correct operation and for possible subsequent reuse. After this definition, the datasets provided by the ``Madrid City Council`` will be transformed to follow these modeling recommendations. With these data sets an application will be developed. It will show the value of this information and its linking, demonstrating the added value of the connection between them and the possibilities that their use could have.
\newline

\textbf{Keywords:} bicycles, Semantic Web, Linked Data, ontology
