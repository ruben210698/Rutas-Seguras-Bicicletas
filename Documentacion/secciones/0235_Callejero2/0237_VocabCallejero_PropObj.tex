\subsection{Propiedades de objeto}
%----------------------------------------------------------------------------------------------------------------------------------------------------------------------------------------------------------------------------------------------
%----------------------------------------------------------------------------------------------------------------------------------------------------------------------------------------------------------------------------------------------
%----------------------------------------------------------------------------------------------------------------------------------------------------------------------------------------------------------------------------------------------
%----------------------------------------------------------------------------------------------------------------------------------------------------------------------------------------------------------------------------------------------




\begin{mybox}{tipoUso}
\begin{flushleft}
\underline{\textbf{IRI:}}
\url{http://vocab.linkeddata.es/datosabiertos/def/urbanismo-infraestructuras/callejero#tipoUso}
\newline

Identificador del tipo de uso que puede tener la calle. Se han definido 2 clases para ello:
\newline -	\url{http://vocab.ciudadesabiertas.es/kos/urbanismo-infraestructuras/calle/tipo-uso/CICLOCALLE}
\newline -	 \url{http://vocab.ciudadesabiertas.es/kos/urbanismo-infraestructuras/calle/tipo-uso/PEATONAL}
\newline


\underline{\textbf{Definida por:}}
\url{http://vocab.linkeddata.es/datosabiertos/def/urbanismo-infraestructuras/callejero}
\newline

\underline{\textbf{Dominio:}} Via
\newline

\underline{\textbf{Rango:}}
	concept
\newline


\end{flushleft}
\end{mybox}
%----------------------------------------------------------------------------------------------------------------------------------------------------------------------------------------------------------------------------------------------





\begin{mybox}{dobleSentido}
\begin{flushleft}
\underline{\textbf{IRI:}}
\url{http://vocab.linkeddata.es/datosabiertos/def/urbanismo-infraestructuras/callejero#dobleSentido}
\newline

Doble sentido o sentido único de una calle.
Puede tomar los siguientes valores definidos en datos.ign.es \cite{datosIgn_calzada}:
%\newline 1: Calle de doble sentido.
%\newline 0: Calle de sentido único.
%\newline { A la espera de conocer su significado oficial, basado en datos aproximados }
\newline \url{http://vocab.ciudadesabiertas.es/kos/urbanismo-infraestructuras/calle/doble-sentido/SENTIDO-UNICO}
\newline \url{http://vocab.ciudadesabiertas.es/kos/urbanismo-infraestructuras/calle/doble-sentido/DOBLE-SENTIDO}
\newline

\underline{\textbf{Definida por:}}\newline
\url{http://vocab.linkeddata.es/datosabiertos/def/urbanismo-infraestructuras/callejero}
\newline

\underline{\textbf{Dominio:}}
		Via
\newline

\underline{\textbf{Rango:}}
 concept
 \newline

\end{flushleft}
\end{mybox}
%----------------------------------------------------------------------------------------------------------------------------------------------------------------------------------------------------------------------------------------------












