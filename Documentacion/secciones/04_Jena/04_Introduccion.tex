\chapter{Generación y búsqueda en ficheros}

Una vez definidos los vocabularios y transformados los datasets acorde con las necesidades que detallaron, se deben generar sus correspondientes ficheros OWL.
Para este proyecto, como viene siendo habitual en la web semántica, se utiliza SPARQL para hacer la búsqueda a través de los conjuntos de datos y utilizar así los recursos. Para ello, en primer lugar, se deben generar los ficheros en formato TTL o RDF, con sus correpondientes separaciones por clases y propiedades de cada uno de sus componentes. Una vez se consiga esta organización de los elementos, se podrán hacer consultas utilizando las queries antes mencionadas y se podrán obtener los elementos requeridos para cada recurso.

Este proceso, como se ha podido observar, se divide en dos secciones bien definidas: la generación de los ficheros en el formato deseado y las búsquedas sobre estos.


\section{Generación de ficheros OWL}

Para esta primera sección se ha utilizado el programa OpenRefine \cite{pagPrinc_OpenRefine}. En él se han cargado los datasets en formato CSV antes transformados y se han generado estos ficheros OWL necesarios para las siguientes consultas.

En este proceso se ha partido de la estructura de los vocabularios definidos anteriormente y se ha replicado con los datos disponibles. 
Para el caso de CallesTranquilas se ha generado un fichero diferente al definido. Dado que no se ha creado un vocabulario específico para ello sino que se han añadido propiedades relativas a ello en el Callejero, no se podía generar del modo en que estaba en el dataset original. Para el vocabulario definido debería existir una propiedad en el dataset de callejero que indicase si una calle es tranquila o no. Dado que dicha propiedad aun no esta creada y que es proporcionada por un conjunto de datos paralelo, se deben obtener esos valores de este último. Para la aplciación que se quiere desarrollar solo es necesario conocer las propiedades de las calles tranquilas y no de todas las vias de Madrid, siguiendo las especificaciones estrictas del vocabulario se tendría que haber obtenido esta propiedad del callejero, sin embargo, se ha tomado la decisión de revisar únicamente el dataset de calles tranquilas, reduciendo asi a un 10$\%$ aproximadamente el total de calles, haciendolo más eficiente y partiendo de la clase via con sus propiedades (como se especifica en la definición de la ontología). En caso de que años posteriores los datasets siguiesen las recomendaciones expuestas en este proyecto sobre la exposición de los datos, se tendrían que hacer ciertas modificaciones en las queries relativas a este datasets, aunque como se ha mencionado, se ha partido del nodo raiz Via y se han consultaso sus propiedades, por lo tanto los cambios serían mínimos.

Para la generación de los ficheros de calles tranquilas se ha tomado como nombre de la via el título original, es decir, la columna relativa al nombre sin corrección de erratas ni eliminación de palabras que provenía de la fuente origen. En el caso de CicloCarriles y Accidentes se ha elegido el nombre de la via. En el primer caso ya que contenía multitud de erratas y se considera más legible el nombre modificado. En el segundo caso debido a que el nombre original contiene el título del cruce en muchos casos, y el nombrado de la vía debe ser único de la misma.

En \ref{cod:ciclocarriles} se ve un ejemplo de la distribución de los elementosdel dataset de Ciclocarriles en el fichero OWL.

\lstinputlisting[style=Python1, label = {cod:ciclocarriles}, caption=ciclocarriles.ttl]{codigo/OWL/ciclocarriles.ttl}


En \ref{cod:callesTranquilas} se ve un ejemplo de la distribución de los elementosdel dataset de Calles Tranquilas en el fichero OWL.

\lstinputlisting[style=Python1, label = {cod:callesTranquilas}, caption=callesTranquilas.ttl]{codigo/OWL/callesTranquilas.ttl}


En \ref{cod:accidentes} se ve un ejemplo de la distribución de los elementos del dataset de Accidentes de bicicletas en el fichero OWL.

\lstinputlisting[style=Python1, label = {cod:accidentes}, caption=accidentes.ttl]{codigo/OWL/accidentes.ttl}







\clearpage
\section{Consultas SPARQL}

Para realizar las búqueda a través de los distintos ficheros Turtle se ha utilizado el lenguaje SPARQL y el framework Apache Jena. En todos los datasets incluidos en este trabajo se han realizado filtros por la via a la que pertenecen para asi determinar otras propiedades relacionadas con esa clase o el propio recurso, del cual posteriormente se pueden obtener sus elementos.


Las siguientes búsquedas son relativas a los ciclocarriles.
\lstinputlisting[style=Python1, label = {cod:ciclocarriles}, caption=Ciclocarriles]{codigo/sparql/ciclocarriles.py}

Como se puede observar se realizan dos consultas. La primera es para conocer las características de la via donde se encuentra, en este caso su nombre y longitud. La segunda es relativa al ciclocarril, del cual en este caso solo se va a consultar si es de uso exclusivo o no.
\newline

Las siguientes búsquedas son relativas a las Calles Tranquilas.
\lstinputlisting[style=Python1, label = {cod:callesTranquilas}, caption=Calles Tranquilas]{codigo/sparql/callesTranquilas.py}

Como se puede observar en esta ocasión solo se realiza una consulta y se hace utilizando la uri de via. Esto es debido a que calle tranquila no es una clase sino una propiedad del Callejero. En este caso no se esta utilizando el callejero de Madrid al completo, solo las calles que sean "tranquilas", sin embargo se sigue el mismo esquema que el definido en la consulta y se debe filtrar por la vía deseada.
En este caso se están obteniendo las propiedades longitud y nombre. En caso de que el Ayuntamiento proporcionase el dataset añadiendo esa propiedad que se ha definido de "calle tranquilas para ciclistas", sería necesario obtenerla. En este caso se filtra por la URI de vía y en caso de que exista es de ese tipo, en caso contrario no.


Las siguientes búsquedas son relativas a los Accidentes.
\lstinputlisting[style=Python1, label = {cod:accidentes}, caption=Accidentes]{codigo/sparql/accidentes.py}

Para el caso de los accidentes se han tenido que realizar 3 búsquedas por cada uno de ellos. En la primera se obtiene la propiedad "nombre" de la vía (para poder mostrar el lugar donde ocurrió). En la segunda se hace una búsqueda del propio accidente. Para ello se filtra por el lugar del hecho, es decir la propiedad "accid:ocurreEnVia". A partir de esta clase se leen las porpiedades hora, lesividad y persona afectada. Esta última se trata de otra clase, como bien se puede ver en los vocabularios antes definidos, por tanto es necesaria una tercera consulta para este recurso. En esta última query se filtra por la URI anterior y se obtiene el tipo de persona afectada, es decir, si es Conductor, Peaton, Acompañante... La división por clases en este dataset permite diferenciar claramente los recursos y sus propiedades y ello obliga a realizar múltiples consultas al fichero de datos.

