\section{Vocabulario de Calles Tranquilas}

Para el vocabulario asociado con las calles tranquilas para ciclistas en el ayuntamiento de Madrid se ha elegido la fuente de Datos.Madrid \cite{datosMadrid_callesTranquilas}.
Proporcionada por el ayuntamiento de Madrid, se muestran las calles más apropiadas para el tránsito de ciclistas. No se dan especificaciones de los criterios utlizados que han llevado a estas calles a formar parte de la lista. Sin embargo, si se puede observar en algnas de ellas ciertos patrones, como que no forman parte de las vias principales de la capital y que son poco transitadas. La misma web ofrece un archivo KML y permite que se puedan mostrar sobre un mapa en Open Street Map \cite{openStreetMapCallesTranquilas}, lo cual proporciona una idea general de su disposición y posibles criterios utilizados.
\newline


Para la representación de los datos de calles tranquilas para ciclistas no se ha definido un modelo sino que se han realizado modificaciones al ya existente para Vias. De esta forma se ha abierto una solicitud al repositorio relativo al vocabulario de Callejero \cite{ciudadesbiertas_callejero} y se han añadido las propiedades necesarias para representar los datos aqui dispuestos. Esta propuesta de modificación se detallará en capitulos posteriores.
\newline
Sin embargo, si se ha hecho uso del dataset proporcionado por el ayuntamiento para la aplicación final que se esta construyendo en el contexto de este trabajo. Se han realizados ciertas modificaciones con respecto al dataset original para que puedan utilizarse sus datos más eficazmente.\newline
Se ha optado por la separación del tipo de via del nombre, conservandola en éste y creando una nueva propiedad que permita saber su clase. Algunos ejemplos serían Calle, Avenida, Plaza...
Se ha añadido el identificador de la via, obtenido a partir del nombre y cruzado con el dataset del callejero de madrid \cite{datosmadrid_callejero}. El identificador permitirá hacer búsquedas mucho más rápidas sobre los datos en caso de querer hacerla filtrando por la calle, que es el caso de la aplicación final que se desea realizar para este proyecto.\newline
En este caso el municipio será siempre Madrid, pero en caso de que se quisiera reulizar en otros proyectos a mayor escala sería necesario conocer la zona geográfica donde se encuentra; por tanto, también se ha añadido, aunque con el valor fijo de Madrid, que corresponde al código 28079, proporcionado por el Instituto Nacional de Estadistica\cite{datosIgnMunicipios}.\newline
Estas transformaciones se detallarán más adelante en la seccion: Transformaciones en los vocabularios.
\newline
Se omite la propiedad ID$\_$TIPO, ya que representa lo mismo que TX$\_$CAPA (el uso que tiene la via) y se ha optado por la segunda por ser representado con texto, más visual y representativo a la hora de su utilización.
\newline


Debido a la falta de disponibilidad de una leyenda o información proporcionada por el ayuntamiento de Madrid, para este conjunto de datos no se han podido conocer con exactitud el significado de sus datos. Ciertos datos no han podido añadirse al modelo por dicho impedimento.


