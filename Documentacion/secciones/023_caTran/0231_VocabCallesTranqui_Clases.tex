\subsection{Clases}



\begin{mybox}{CalleTranquila}
\begin{flushleft}
\underline{\textbf{IRI:}}
\url{http://vocab.ciudadesabiertas.es/def/calle-bici/calle-tranquila#CalleTranquila}
\newline

Calle apropiada para el tránsito de ciclistas siguiendo los cirterios del ayuntamiento de Madrid.
\newline

\underline{\textbf{Definida por:}}
\url{http://vocab.ciudadesabiertas.es/def/calle-bici/calle-tranquila}
\newline

\underline{\textbf{Tiene subclase:}}
\newline idObject,\hspace{2em}  idGrupo,\hspace{2em} idEje,	\hspace{2em} tipoUso,
\newline municipio,\hspace{2em} Calle

\underline{\textbf{Tiene Superclase:}}
\newline longitud

\end{flushleft}
\end{mybox}

%----------------------------------------------------------------------------------------------------------------------------------------------------------------------------------------------------------------------------------------------






\begin{mybox}{Calle}
\begin{flushleft}
\underline{\textbf{IRI:}}
\url{http://vocab.ciudadesabiertas.es/def/calle-bici/calle-tranquila#Calle}
\newline

Representación de una via de una ciudad.
\newline

\underline{\textbf{Definida por:}}
\url{http://vocab.ciudadesabiertas.es/def/calle-bici/calle-tranquila}
\newline

\underline{\textbf{Tiene subclase:}}
\newline longitud,\hspace{2em}  nuDireccion,
\newline calzada,\hspace{2em} tipoVia

\underline{\textbf{Tiene Superclases:}}
\newline CalleTranquila
\newline nombre oficial

\end{flushleft}
\end{mybox}

%----------------------------------------------------------------------------------------------------------------------------------------------------------------------------------------------------------------------------------------------






\begin{mybox}{Via}
\begin{flushleft}
\underline{\textbf{IRI:}}
\url{http://vocab.linkeddata.es/datosabiertos/def/urbanismo-infraestructuras/callejero#Via}
\newline

Se ha reutilizado la definición de Municipio proporcionada por vocab.linkeddata.es \cite{datoabiertos_idVia}

Vía de comunicación construida para la circulación. En su definición según el modelo de direcciones de la Administración General del Estado, Incluye calles, carreteras de todo tipo, caminos, vías de agua, pantalanes, etc. Asimismo, incluye la pseudovía., es decir todo aquello que complementa o sustituye a la vía. En nuestro caso, este término se utiliza para hacer referencia a las vías urbanas.
Representación numérica de la misma.
\newline

\underline{\textbf{Definida por:}}
\url{http://vocab.linkeddata.es/datosabiertos/def/urbanismo-infraestructuras/callejero}
\newline

\underline{\textbf{Tiene Superclases:}}
\newline Calle





\end{flushleft}
\end{mybox}
%----------------------------------------------------------------------------------------------------------------------------------------------------------------------------------------------------------------------------------------------




\begin{mybox}{municipio}
\begin{flushleft}
\underline{\textbf{IRI:}}
\url{http://vocab.linkeddata.es/datosabiertos/def/sector-publico/territorio#Municipio}
\newline

Se ha reutilizado la definición de Municipio proporcionada por vocab.linkeddata.es \cite{datoabiertos_municipio}
Un Municipio es el ente local definido en el artículo 140 de la Constitución española y la entidad básica de la organización territorial del Estado según el artículo 1 de la Ley 7/1985, de 2 de abril, Reguladora de las Bases del Régimen Local. Tiene personalidad jurídica y plena capacidad para el cumplimiento de sus fines. La delimitación territorial de Municipio está recogida del REgistro Central de Cartografía del IGN. Su nombre, que se especifica con la propiedad dct:title, es el proporcionado por el Registro de Entidades Locales del Ministerio de Política Territorial, en \url{http://www.ine.es/nomen2/index.do}
\newline


\underline{\textbf{Definida por:}}
\url{http://purl.org/derecho/vocabulario}
\url{http://vocab.linkeddata.es/datosabiertos/def/sector-publico/territorio}
\url{http://www.ign.es/ign/resources/acercaDe/tablon/ModeloDireccionesAGE}
\newline

\underline{\textbf{Tiene Superclases:}}
\newline CalleTranquila



\end{flushleft}
\end{mybox}
%----------------------------------------------------------------------------------------------------------------------------------------------------------------------------------------------------------------------------------------------





\begin{mybox}{idObject}
\begin{flushleft}
\underline{\textbf{IRI:}}
\url{http://vocab.ciudadesabiertas.es/def/calle-bici/calle-tranquila#idObject}
\newline

Identificador propio del ayuntamiento.
%\newline{A la espera de conocer su significado}
\newline


\underline{\textbf{Definida por:}}
\url{http://vocab.ciudadesabiertas.es/def/calle-bici/calle-tranquila}
\newline

\underline{\textbf{Tiene Superclases:}}
\newline CalleTranquila

\end{flushleft}
\end{mybox}
%----------------------------------------------------------------------------------------------------------------------------------------------------------------------------------------------------------------------------------------------




\begin{mybox}{idGrupo}
\begin{flushleft}
\underline{\textbf{IRI:}}
\url{http://vocab.ciudadesabiertas.es/def/calle-bici/calle-tranquila#idGrupo}
\newline

Identificador propio del ayuntamiento. 
%\newline{A la espera de conocer su significado}
\newline

\underline{\textbf{Definida por:}}
\url{http://vocab.ciudadesabiertas.es/def/calle-bici/calle-tranquila}
\newline

\underline{\textbf{Tiene Superclases:}}
\newline CalleTranquila

\end{flushleft}
\end{mybox}
%----------------------------------------------------------------------------------------------------------------------------------------------------------------------------------------------------------------------------------------------




\begin{mybox}{idEje}
\begin{flushleft}
\underline{\textbf{IRI:}}
\url{http://vocab.ciudadesabiertas.es/def/calle-bici/calle-tranquila#idEje}
\newline

Identificador propio del ayuntamiento. 
%\newline{A la espera de conocer su significado}
\newline


\underline{\textbf{Definida por:}}
\url{http://vocab.ciudadesabiertas.es/def/calle-bici/calle-tranquila}
\newline

\underline{\textbf{Tiene Superclases:}}
\newline CalleTranquila

\end{flushleft}
\end{mybox}
%----------------------------------------------------------------------------------------------------------------------------------------------------------------------------------------------------------------------------------------------







%----------------------------------------------------------------------------------------------------------------------------------------------------------------------------------------------------------------------------------------------
%----------------------------------------------------------------------------------------------------------------------------------------------------------------------------------------------------------------------------------------------
%----------------------------------------------------------------------------------------------------------------------------------------------------------------------------------------------------------------------------------------------
%----------------------------------------------------------------------------------------------------------------------------------------------------------------------------------------------------------------------------------------------


