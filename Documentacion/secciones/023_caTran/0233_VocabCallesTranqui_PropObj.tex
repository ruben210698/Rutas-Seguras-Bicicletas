\subsection{Propiedades de objeto}
%----------------------------------------------------------------------------------------------------------------------------------------------------------------------------------------------------------------------------------------------
%----------------------------------------------------------------------------------------------------------------------------------------------------------------------------------------------------------------------------------------------
%----------------------------------------------------------------------------------------------------------------------------------------------------------------------------------------------------------------------------------------------
%----------------------------------------------------------------------------------------------------------------------------------------------------------------------------------------------------------------------------------------------



\begin{mybox}{tieneCalle}
\begin{flushleft}
\underline{\textbf{IRI:}}
\url{http://vocab.ciudadesabiertas.es/def/calle-bici/calle-tranquila#tieneCalle}
\newline

Calle asociada a una Calle Tranquila.
\newline

\underline{\textbf{Definida por:}}
\url{http://vocab.ciudadesabiertas.es/def/calle-bici/calle-tranquila}
\newline

\underline{\textbf{Dominio:}}
		CalleTranquila
\newline

\underline{\textbf{Rango:}}
		Calle
\newline


\end{flushleft}
\end{mybox}
%----------------------------------------------------------------------------------------------------------------------------------------------------------------------------------------------------------------------------------------------



\begin{mybox}{tipoUso}
\begin{flushleft}
\underline{\textbf{IRI:}}
\url{http://vocab.ciudadesabiertas.es/def/calle-bici/calle-tranquila#tipoUso}
\newline

Identificador del tipo de uso que puede tener la calle. Se han definido 2 clases para ello:
\newline -	\url{http://vocab.ciudadesabiertas.es/kos/urbanismo-infraestructuras/calle/tipo-uso/CICLOCALLE}
\newline -	 \url{http://vocab.ciudadesabiertas.es/kos/urbanismo-infraestructuras/calle/tipo-uso/PEATONAL}
\newline


%\underline{\textbf{Definida por:}}
%\url{http://vocab.ciudadesabiertas.es/def/calle-bici}
%\newline

\underline{\textbf{Dominio:}}
\newline CalleTranquila

\underline{\textbf{Rango:}}
		concept


\end{flushleft}
\end{mybox}
%----------------------------------------------------------------------------------------------------------------------------------------------------------------------------------------------------------------------------------------------



\begin{mybox}{tieneIdEje}
\begin{flushleft}
\underline{\textbf{IRI:}}
\url{http://vocab.ciudadesabiertas.es/def/calle-bici/calle-tranquila#tieneIdEje}
\newline

Identificador de Eje asociado a una calle.
\newline

\underline{\textbf{Definida por:}}
\url{http://vocab.ciudadesabiertas.es/def/calle-bici/calle-tranquila}
\newline

\underline{\textbf{Dominio:}}
		CalleTranquila
\newline

\underline{\textbf{Rango:}}
		idEje

\end{flushleft}
\end{mybox}
%----------------------------------------------------------------------------------------------------------------------------------------------------------------------------------------------------------------------------------------------





\begin{mybox}{tieneIdGrupo}
\begin{flushleft}
\underline{\textbf{IRI:}}
\url{http://vocab.ciudadesabiertas.es/def/calle-bici/calle-tranquila#tieneIdGrupo}
\newline

Identificador de Grupo asociado a una calle.
\newline

\underline{\textbf{Definida por:}}
\url{http://vocab.ciudadesabiertas.es/def/calle-bici/calle-tranquila}
\newline

\underline{\textbf{Dominio:}}
		CalleTranquila
\newline

\underline{\textbf{Rango:}}
		idGrupo

\end{flushleft}
\end{mybox}
%----------------------------------------------------------------------------------------------------------------------------------------------------------------------------------------------------------------------------------------------




\begin{mybox}{tieneIdObject}
\begin{flushleft}
\underline{\textbf{IRI:}}
\url{http://vocab.ciudadesabiertas.es/def/calle-bici/calle-tranquila#tieneIdObject}
\newline

Identificador de Objeto asociado a una calle.
\newline

\underline{\textbf{Definida por:}}
\url{http://vocab.ciudadesabiertas.es/def/calle-bici/calle-tranquila}
\newline

\underline{\textbf{Dominio:}}
		CalleTranquila
\newline

\underline{\textbf{Rango:}}
		idObject

\end{flushleft}
\end{mybox}
%----------------------------------------------------------------------------------------------------------------------------------------------------------------------------------------------------------------------------------------------






\begin{mybox}{tieneIdVia}
\begin{flushleft}
\underline{\textbf{IRI:}}
\url{http://vocab.ciudadesabiertas.es/def/calle-bici/calle-tranquila#tieneIdVia}
\newline

Identificador de calle asociado.
\newline

\underline{\textbf{Definida por:}}
\url{http://vocab.ciudadesabiertas.es/def/calle-bici/calle-tranquila}
\newline

\underline{\textbf{Dominio:}}
		Calle
\newline

\underline{\textbf{Rango:}}
		Via
\newline


\end{flushleft}
\end{mybox}
%----------------------------------------------------------------------------------------------------------------------------------------------------------------------------------------------------------------------------------------------



\begin{mybox}{municipio}
\begin{flushleft}
\underline{\textbf{IRI:}}
\url{http://vocab.linkeddata.es/datosabiertos/def/sector-publico/territorio#municipio}
\newline

Municipio al que pertenece un fenómeno geográfico o una entidad administrativa.
\newline

\underline{\textbf{Definida por:}}
\url{http://vocab.linkeddata.es/datosabiertos/def/sector-publico/territorio}
\newline

\underline{\textbf{Dominio:}}
		CalleTranquila
\newline

\underline{\textbf{Rango:}}
		Municipio

\end{flushleft}
\end{mybox}



%----------------------------------------------------------------------------------------------------------------------------------------------------------------------------------------------------------------------------------------------





\begin{mybox}{nuDireccion}
\begin{flushleft}
\underline{\textbf{IRI:}}
\url{http://vocab.ciudadesabiertas.es/def/calle-bici#nuDireccion}
\newline

Dirección y sentido de la calle o tramo de la calle a la que se refiere.
Puede tomar los siguientes valores:
\newline \url{http://vocab.ciudadesabiertas.es/recurso/calle-bici#SOL} - Dirección y sentido a Sol
\newline \url{http://vocab.ciudadesabiertas.es/recurso/calle-bici#paraleloSOL} - Dirección paralela a Sol
\newline \url{http://vocab.ciudadesabiertas.es/recurso/calle-bici#contrarioSOL} - Dirección a sol y sentido opuesto.
\newline

Esta definición esta basada en datos aproximados proporcionados por mapas y otros elementos que han podido verse en los ejemplos del dataset. Pueden ser erróneos o no reflejar con exactitud la realidad del dato ya que no existe una leyenda que poder consultar.
\newline

%\newline http://vocab.ciudadesabiertas.es/recurso/calle-bici#1: Dirección y sentido a Sol.
%\newline http://vocab.ciudadesabiertas.es/recurso/calle-bici#0: Dirección paralela a cualquiera que apunte a Sol.
%\newline http://vocab.ciudadesabiertas.es/recurso/calle-bici#-1: Dirección a Sol, sentido opuesto.
%\newline { A la espera de conocer su significado oficial, basado en datos aproximados }


\underline{\textbf{Definida por:}}
\url{http://vocab.ciudadesabiertas.es/def/calle-bici}
\newline

\underline{\textbf{Dominio:}}
		Calle
\newline

\underline{\textbf{Rango}}
\newline concept

\end{flushleft}
\end{mybox}
%----------------------------------------------------------------------------------------------------------------------------------------------------------------------------------------------------------------------------------------------






\begin{mybox}{calzada}
\begin{flushleft}
\underline{\textbf{IRI:}}
\url{https://datos.ign.es/def/btn100#calzada}
\newline

Doble sentido o sentido único de una calle.
Puede tomar los siguientes valores definidos en datos.ign.es \cite{datosIgn_calzada}:
%\newline 1: Calle de doble sentido.
%\newline 0: Calle de sentido único.
%\newline { A la espera de conocer su significado oficial, basado en datos aproximados }
\newline \url{https://datos.ign.es/kos/transportes/calzada/convencional}
\newline \url{https://datos.ign.es/kos/transportes/calzada/doble}
\newline \url{https://datos.ign.es/kos/transportes/calzada/sentido-unico}
\newline

\underline{\textbf{Definida por:}}
\url{https://datos.ign.es/kos/transportes/calzada/}
\newline

\underline{\textbf{Dominio:}}
		Calle
\newline

\underline{\textbf{Rango:}}
\newline concept

\end{flushleft}
\end{mybox}
%----------------------------------------------------------------------------------------------------------------------------------------------------------------------------------------------------------------------------------------------














\begin{mybox}{tipoVia}
\begin{flushleft}
\underline{\textbf{IRI:}}
\url{http://vocab.linkeddata.es/datosabiertos/def/urbanismo-infraestructuras/callejero#tipoVia}
\newline

Se ha reutilizado la definición de tipoVia proporcionada por vocab.linkeddata.es \cite{datoabiertos_tipoVia}.
Tipo de vía, que será representado mediante la clasificación en SKOS de URI \url{http://vocab.linkeddata.es/datosabiertos/kos/urbanismo-infraestructuras/tipo-via}. Por ejemplo, estas serán las URIs correspondientes a calles y plazas \url{http://vocab.linkeddata.es/datosabiertos/kos/urbanismo-infraestructuras/tipo-via/CL} \url{http://vocab.linkeddata.es/datosabiertos/kos/urbanismo-infraestructuras/tipo-via/PL}
\newline

\underline{\textbf{Definida por:}}
\url{http://vocab.linkeddata.es/datosabiertos/def/urbanismo-infraestructuras/callejero}
\newline

\underline{\textbf{Dominio:}}
		Calle
\newline

\underline{\textbf{Rango:}}
		concept

\end{flushleft}
\end{mybox}
%----------------------------------------------------------------------------------------------------------------------------------------------------------------------------------------------------------------------------------------------























