\subsection{Clases}



\begin{mybox}{Accidente}
\begin{flushleft}
\underline{\textbf{IRI:}}

\url{http://vocab.ciudadesabiertas.es/def/transporte/accidente#Accidente}
\newline

Siniestro ocurrido en la vía pública con implicación de algún vehículo. El recurso se construirá a partir de su número de expediente.
\newline

\underline{\textbf{Definida por:}}\newline
\url{http://vocab.ciudadesabiertas.es/def/transporte/accidente}
\newline




\end{flushleft}
\end{mybox}

%----------------------------------------------------------------------------------------------------------------------------------------------------------------------------------------------------------------------------------------------

\begin{mybox}{PersonaAfectada}
\begin{flushleft}
\underline{\textbf{IRI:}}

\url{http://vocab.ciudadesabiertas.es/def/transporte/accidente#PersonaAfectada}
\newline

La persona perjudicada por el accidente de tráfico.
\newline

\underline{\textbf{Definida por:}}
\newline
\url{http://vocab.ciudadesabiertas.es/def/transporte/accidente}
\newline

\end{flushleft}
\end{mybox}
%----------------------------------------------------------------------------------------------------------------------------------------------------------------------------------------------------------------------------------------------


\begin{mybox}{Portal}
\begin{flushleft}
\underline{\textbf{IRI:}}
\url{http://vocab.linkeddata.es/datosabiertos/def/urbanismo-infraestructuras/callejero#Portal}
\newline

Ha sido definido por la plataforma ciudadesabiertas \cite{datosabiertos_portal}.
Subacceso independiente exterior (al aire libre) a una misma construcción. Para una misma construcción, con un mismo número de vía, pueden existir varias entradas que pueden estar numeradas con números o letras. [fuente: Modelo de Direcciones de la Administración General del Estado v.2]
\newline

\underline{\textbf{Definida por:}}\newline
\url{http://vocab.linkeddata.es/datosabiertos/def/urbanismo-infraestructuras/callejero}
\newline


\end{flushleft}
\end{mybox}
%----------------------------------------------------------------------------------------------------------------------------------------------------------------------------------------------------------------------------------------------


\begin{mybox}{Municipio}
\begin{flushleft}
\underline{\textbf{IRI:}}
\url{http://vocab.linkeddata.es/datosabiertos/def/sector-publico/territorio#Municipio}
\newline

Se ha reutilizado la definición de Municipio proporcionada por vocab.linkeddata.es \cite{datoabiertos_municipio}
Un Municipio es el ente local definido en el artículo 140 de la Constitución española y la entidad básica de la organización territorial del Estado según el artículo 1 de la Ley 7/1985, de 2 de abril, Reguladora de las Bases del Régimen Local. Tiene personalidad jurídica y plena capacidad para el cumplimiento de sus fines. La delimitación territorial de Municipio está recogida del Registro Central de Cartografía del IGN. Su nombre, que se especifica con la propiedad dct:title, es el proporcionado por el Registro de Entidades Locales del Ministerio de Política Territorial, en \url{http://www.ine.es/nomen2/index.do}
\newline


\underline{\textbf{Definida por:}}
\newline \url{http://purl.org/derecho/vocabulario}
\newline \url{http://vocab.linkeddata.es/datosabiertos/def/sector-publico/territorio}
\newline \url{http://www.ign.es/ign/resources/acercaDe/tablon/ModeloDireccionesAGE}
\newline

\underline{\textbf{Esta en rango de:}} municipio



\end{flushleft}
\end{mybox}
%----------------------------------------------------------------------------------------------------------------------------------------------------------------------------------------------------------------------------------------------





\begin{mybox}{Via}
\begin{flushleft}
\underline{\textbf{IRI:}}
\url{http://vocab.linkeddata.es/datosabiertos/def/urbanismo-infraestructuras/callejero#Via}
\newline

Se ha reutilizado la definición de Municipio proporcionada por vocab.linkeddata.es \cite{datoabiertos_idVia}

Vía de comunicación construida para la circulación. En su definición según el modelo de direcciones de la Administración General del Estado, Incluye calles, carreteras de todo tipo, caminos, vías de agua, pantanales, etc. Asimismo, incluye la pseudovía., es decir todo aquello que complementa o sustituye a la vía. En nuestro caso, este término se utiliza para hacer referencia a las vías urbanas.
Representación numérica de la misma.
\newline

\underline{\textbf{Definida por:}}\newline
\url{http://vocab.linkeddata.es/datosabiertos/def/urbanismo-infraestructuras/callejero}
\newline




\end{flushleft}
\end{mybox}
%----------------------------------------------------------------------------------------------------------------------------------------------------------------------------------------------------------------------------------------------





\begin{mybox}{Gender}
\begin{flushleft}
\underline{\textbf{IRI:}}
\url{https://schema.org/gender}
\newline

Género de la persona afectada.
\\Seguirá el formato definido por Schema.org
\\Se utilizarán las siguientes definidas en la clase:
\\http://schema.org/Male
\\http://schema.org/Female
\\http://schema.org/Mixed
\newline

\underline{\textbf{Definida por:}}\newline
\url{https://schema.org/gender}
\newline




\end{flushleft}
\end{mybox}
%----------------------------------------------------------------------------------------------------------------------------------------------------------------------------------------------------------------------------------------------



%----------------------------------------------------------------------------------------------------------------------------------------------------------------------------------------------------------------------------------------------
%----------------------------------------------------------------------------------------------------------------------------------------------------------------------------------------------------------------------------------------------
%----------------------------------------------------------------------------------------------------------------------------------------------------------------------------------------------------------------------------------------------
%----------------------------------------------------------------------------------------------------------------------------------------------------------------------------------------------------------------------------------------------


