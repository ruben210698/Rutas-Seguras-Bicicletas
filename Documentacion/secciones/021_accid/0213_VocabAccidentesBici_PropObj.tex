\subsection{Propiedades de objeto}
%----------------------------------------------------------------------------------------------------------------------------------------------------------------------------------------------------------------------------------------------
%----------------------------------------------------------------------------------------------------------------------------------------------------------------------------------------------------------------------------------------------
%----------------------------------------------------------------------------------------------------------------------------------------------------------------------------------------------------------------------------------------------
%----------------------------------------------------------------------------------------------------------------------------------------------------------------------------------------------------------------------------------------------




\begin{mybox}{hasPersonaAfectada}
\begin{flushleft}
\underline{\textbf{IRI:}}
\url{http://vocab.ciudadesabiertas.es/def/transporte/accidente#hasPersonaAfectada}
\newline

Persona que se asocia a un accidente. Esta a su vez puede tener más características como por ejemplo el rol que tuvo (peatón, conductor), edad y género.
\newline

\underline{\textbf{Definida por:}}
\newline \url{http://vocab.ciudadesabiertas.es/def/transporte/accidente}
\newline

\underline{\textbf{Dominio:}} 	
\newline Accidente
\newline

\underline{\textbf{Rango:}} 
\newline PersonaAfectada

\end{flushleft}
\end{mybox}
%----------------------------------------------------------------------------------------------------------------------------------------------------------------------------------------------------------------------------------------------




\begin{mybox}{tipoVehiculo}
\begin{flushleft}
\underline{\textbf{IRI:}}
\url{http://vocab.ciudadesabiertas.es/def/transporte/accidente#tipoVehiculo}
\newline

Tipo de vehículo afectado, p.ej. Bicicleta, Bicicleta EPAC (pedaleo asistido). Se han definido los siguientes elementos:
\newline \url{http://vocab.linkeddata.es/datosabiertos/kos/transporte/accidente/tipo-vehiculo/BICICLETA}
\newline \url{http://vocab.linkeddata.es/datosabiertos/kos/transporte/accidente/tipo-vehiculo/BICICLETA-EPAC}
\newline

\underline{\textbf{Definida por:}}\newline
\url{http://vocab.ciudadesabiertas.es/def/transporte/accidente}
\newline

\underline{\textbf{Dominio:}} Accidente
\newline

\underline{\textbf{Rango:}} concept
\newline

\end{flushleft}
\end{mybox}
%----------------------------------------------------------------------------------------------------------------------------------------------------------------------------------------------------------------------------------------------



\begin{mybox}{meteorologia}
\begin{flushleft}
\underline{\textbf{IRI:}}
\url{http://vocab.ciudadesabiertas.es/def/transporte/accidente#meteorologia}
\newline

Condiciones ambientales que se dan en el momento del siniestro. Se han definido varios tipos posibles:
\newline \url{http://vocab.linkeddata.es/datosabiertos/kos/transporte/accidente/meteorologia/DESPEJADO}
\newline \url{http://vocab.linkeddata.es/datosabiertos/kos/transporte/accidente/meteorologia/LLUVIA-DEBIL}
\newline \url{http://vocab.linkeddata.es/datosabiertos/kos/transporte/accidente/meteorologia/LLUVIA-INTENSA}
\newline \url{http://vocab.linkeddata.es/datosabiertos/kos/transporte/accidente/meteorologia/NUBLADO}
\newline \url{http://vocab.linkeddata.es/datosabiertos/kos/transporte/accidente/meteorologia/GRANIZANDO}
\newline \url{http://vocab.linkeddata.es/datosabiertos/kos/transporte/accidente/meteorologia/DESCONOCIDO}
\newline

\underline{\textbf{Definida por:}}\newline
\url{http://vocab.ciudadesabiertas.es/def/transporte/accidente}
\newline

\underline{\textbf{Dominio:}}  Accidente
\newline

\underline{\textbf{Rango:}} concept
\newline

\end{flushleft}
\end{mybox}
%----------------------------------------------------------------------------------------------------------------------------------------------------------------------------------------------------------------------------------------------


\begin{mybox}{tipoAccidente}
\begin{flushleft}
\underline{\textbf{IRI:}}
\url{http://vocab.ciudadesabiertas.es/def/transporte/accidente#tipoAccidente}
\newline

Tipo de accidente asociado. Se han definido para ello varios tipos posibles:
\newline \url{http://vocab.linkeddata.es/datosabiertos/kos/transporte/accidente/tipo-accidente/COLISION}
\newline \url{http://vocab.linkeddata.es/datosabiertos/kos/transporte/accidente/tipo-accidente/COLISION-DOBLE}
\newline \url{http://vocab.linkeddata.es/datosabiertos/kos/transporte/accidente/tipo-accidente/COLISION-MULTIPLE}
\newline \url{http://vocab.linkeddata.es/datosabiertos/kos/transporte/accidente/tipo-accidente/ALCANCE}
\newline \url{http://vocab.linkeddata.es/datosabiertos/kos/transporte/accidente/tipo-accidente/CHOQUE-NO-VEHICULO}
\newline \url{http://vocab.linkeddata.es/datosabiertos/kos/transporte/accidente/tipo-accidente/ATROPELLO-PEATON}
\newline \url{http://vocab.linkeddata.es/datosabiertos/kos/transporte/accidente/tipo-accidente/VUELCO}
\newline \url{http://vocab.linkeddata.es/datosabiertos/kos/transporte/accidente/tipo-accidente/CAIDA}
\newline \url{http://vocab.linkeddata.es/datosabiertos/kos/transporte/accidente/tipo-accidente/OTROS}
\newline

\underline{\textbf{Definida por:}}
\newline \url{http://vocab.ciudadesabiertas.es/def/transporte/accidente}
\newline

\underline{\textbf{Dominio:}}  Accidente
\newline

\underline{\textbf{Rango:}}  concept
\newline

\end{flushleft}
\end{mybox}
%----------------------------------------------------------------------------------------------------------------------------------------------------------------------------------------------------------------------------------------------




\begin{mybox}{lesividad}
\begin{flushleft}
\underline{\textbf{IRI:}}
\url{http://vocab.ciudadesabiertas.es/def/transporte/accidente#lesividad}
\newline

Código que indica la gravedad del siniestro para la persona afectada.
\newline

Para su uso se han definido los siguientes elementos:
\newline 01 Atencion en urgencias sin posterior ingreso. - LEVE:
\newline \url{http://vocab.linkeddata.es/datosabiertos/kos/transporte/accidente/lesividad/01}
\newline 02 Ingreso inferior o igual a 24 horas - LEVE:
\newline \url{http://vocab.linkeddata.es/datosabiertos/kos/transporte/accidente/lesividad/02}
\newline 03 Ingreso superior a 24 horas. - GRAVE:
\newline \url{http://vocab.linkeddata.es/datosabiertos/kos/transporte/accidente/lesividad/03}
\newline 04 Fallecido 24 horas - FALLECIDO:
\newline \url{http://vocab.linkeddata.es/datosabiertos/kos/transporte/accidente/lesividad/04}
\newline 05 Asistencia sanitaria ambulatoria con posterioridad - LEVE:
\newline \url{http://vocab.linkeddata.es/datosabiertos/kos/transporte/accidente/lesividad/05}
\newline 06 Asistencia sanitaria inmediata en centro de salud o mutua - LEVE:
\newline \url{http://vocab.linkeddata.es/datosabiertos/kos/transporte/accidente/lesividad/06}
\newline 07 Asistencia sanitaria solo en el lugar del accidente - LEVE:
\newline \url{http://vocab.linkeddata.es/datosabiertos/kos/transporte/accidente/lesividad/07}
\newline 14 Sin asistencia sanitaria:
\newline \url{http://vocab.linkeddata.es/datosabiertos/kos/transporte/accidente/lesividad/14}
\newline 77 Se desconoce:
\newline \url{http://vocab.linkeddata.es/datosabiertos/kos/transporte/accidente/lesividad/77}
\newline


\underline{\textbf{Definida por:}}
\newline \url{http://vocab.ciudadesabiertas.es/def/transporte/accidente}
\newline

\underline{\textbf{Dominio:}}  Accidente
\newline

\underline{\textbf{Rango:}} concept
\newline

\end{flushleft}
\end{mybox}
%----------------------------------------------------------------------------------------------------------------------------------------------------------------------------------------------------------------------------------------------






\begin{mybox}{gender}
\begin{flushleft}
\underline{\textbf{IRI:}}
\url{https://schema.org/gender}
\newline

Género de la persona afectada.
\newline Seguirá el formato definido por Schema.org \cite{schema_gender}
Se utilizarán las siguientes definidas en la clase:
\newline \url{http://schema.org/Male}
\newline \url{http://schema.org/Female}
\newline \url{http://schema.org/Mixed}
\newline

\underline{\textbf{Definida por:}}\newline
\url{https://schema.org/gender}
\newline

\underline{\textbf{Dominio:}} PersonaAfectada
\newline

\underline{\textbf{Rango:}} Gender \cite{schema_gender_explicacion_rango}

\end{flushleft}
\end{mybox}
%----------------------------------------------------------------------------------------------------------------------------------------------------------------------------------------------------------------------------------------------







\begin{mybox}{tipoPersAfect}
\begin{flushleft}
\underline{\textbf{IRI:}}
\url{http://vocab.ciudadesabiertas.es/def/transporte/accidente#tipoPersAfect}
\newline

Persona a la que afecta el accidente. Puede ser Conductor, peatón, testigo o viajero. Se han definido los siguientes elementos:
\newline \url{http://vocab.linkeddata.es/datosabiertos/kos/transporte/accidente/tipo-pers-afect/CONDUCTOR}
\newline \url{http://vocab.linkeddata.es/datosabiertos/kos/transporte/accidente/tipo-pers-afect/PEATON}
\newline \url{http://vocab.linkeddata.es/datosabiertos/kos/transporte/accidente/tipo-pers-afect/TESTIGO}
\newline \url{http://vocab.linkeddata.es/datosabiertos/kos/transporte/accidente/tipo-pers-afect/VIAJERO}
\newline

\underline{\textbf{Definida por:}}\newline
\url{http://vocab.ciudadesabiertas.es/def/transporte/accidente}
\newline

\underline{\textbf{Dominio:}} PersonaAfectada
\newline

\underline{\textbf{Rango:}} concept
\newline

\end{flushleft}
\end{mybox}
%----------------------------------------------------------------------------------------------------------------------------------------------------------------------------------------------------------------------------------------------














\begin{mybox}{portal}
\begin{flushleft}
\underline{\textbf{IRI:}}
\url{http://vocab.linkeddata.es/datosabiertos/def/urbanismo-infraestructuras/callejero#portal}
\newline

Numero de la calle donde ha ocurrido el accidente, si procede.
\newline

\underline{\textbf{Definida por:}}\newline
\url{http://vocab.linkeddata.es/datosabiertos/def/urbanismo-infraestructuras/callejero}
\newline

\underline{\textbf{Dominio:}} Accidente
\newline

\underline{\textbf{Rango:}} Portal
\newline

\end{flushleft}
\end{mybox}
%----------------------------------------------------------------------------------------------------------------------------------------------------------------------------------------------------------------------------------------------










\begin{mybox}{ocurreEnVia}
\begin{flushleft}
\underline{\textbf{IRI:}}
\url{http://vocab.ciudadesabiertas.es/def/transporte/accidente#ocurreEnVia}
\newline

Propiedad que permite conocer las vías asociadas a un accidente. Puede haber varias en el caso de que haya ocurrido en un cruce.

\underline{\textbf{Definida por:}}\newline
\url{http://vocab.ciudadesabiertas.es/def/transporte/accidente}
\newline

\underline{\textbf{Dominio:}} Accidente
\newline

\underline{\textbf{Rango:}} Via
\newline

\end{flushleft}
\end{mybox}
%----------------------------------------------------------------------------------------------------------------------------------------------------------------------------------------------------------------------------------------------





\begin{mybox}{municipio}
\begin{flushleft}
\underline{\textbf{IRI:}}
\url{http://vocab.linkeddata.es/datosabiertos/def/sector-publico/territorio#municipio}
\newline

Municipio al que pertenece un fenómeno geográfico o una entidad administrativa  \cite{datoabiertos_municipio}.
\newline

\underline{\textbf{Definida por:}}\newline
\url{http://vocab.linkeddata.es/datosabiertos/def/sector-publico/territorio}
\newline

\underline{\textbf{Dominio:}}		Via
\newline

\underline{\textbf{Rango:}}		Municipio

\end{flushleft}
\end{mybox}



%----------------------------------------------------------------------------------------------------------------------------------------------------------------------------------------------------------------------------------------------






\begin{mybox}{portal}
\begin{flushleft}
\underline{\textbf{IRI:}}
\url{http://vocab.linkeddata.es/datosabiertos/def/urbanismo-infraestructuras/callejero#portal}
\newline

Portal asociado a un accidente.
\newline

\underline{\textbf{Definida por:}}\newline
\url{http://vocab.linkeddata.es/datosabiertos/def/urbanismo-infraestructuras/callejero}
\newline

\underline{\textbf{Dominio:}}		Accidente
\newline

\underline{\textbf{Rango:}}		Via

\end{flushleft}
\end{mybox}



%----------------------------------------------------------------------------------------------------------------------------------------------------------------------------------------------------------------------------------------------






























