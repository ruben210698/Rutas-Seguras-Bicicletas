\subsection{Propiedades de objeto}
%----------------------------------------------------------------------------------------------------------------------------------------------------------------------------------------------------------------------------------------------
%----------------------------------------------------------------------------------------------------------------------------------------------------------------------------------------------------------------------------------------------
%----------------------------------------------------------------------------------------------------------------------------------------------------------------------------------------------------------------------------------------------
%----------------------------------------------------------------------------------------------------------------------------------------------------------------------------------------------------------------------------------------------




\begin{mybox}{tienePersonaAfectada}
\begin{flushleft}
\underline{\textbf{IRI:}}
\url{http://vocab.ciudadesabiertas.es/def/accidente/accid-bici#tienePersonaAfectada}
\newline

Persona que se asocia a un accidente.
\newline

\underline{\textbf{Definida por:}}
\newline \url{http://vocab.ciudadesabiertas.es/def/accidente/accid-bici}
\newline

\underline{\textbf{Dominio:}} 	
\newline Accidente
\newline

\underline{\textbf{Rango:}} 
\newline PersonaAfectada

\end{flushleft}
\end{mybox}
%----------------------------------------------------------------------------------------------------------------------------------------------------------------------------------------------------------------------------------------------




\begin{mybox}{tieneLugarAccidente}
\begin{flushleft}
\underline{\textbf{IRI:}}
\url{http://vocab.ciudadesabiertas.es/def/accidente/accid-bici#tieneLugarAccidente}
\newline

Lugar de los acontecimientos asociados a un accidente.
\newline

\underline{\textbf{Definida por:}}
\newline \url{http://vocab.ciudadesabiertas.es/def/accidente/accid-bici}
\newline

\underline{\textbf{Dominio:}} 
\newline Accidente
\newline

\underline{\textbf{Rango:}} 
\newline LugarAccidente
\newline

\end{flushleft}
\end{mybox}
%----------------------------------------------------------------------------------------------------------------------------------------------------------------------------------------------------------------------------------------------



\begin{mybox}{tieneCalle}
\begin{flushleft}
\underline{\textbf{IRI:}}
\url{http://vocab.ciudadesabiertas.es/def/accidente/accid-bici#tieneCalle}
\newline

Calle asociada al lugar donde ocurrió el accidente.
\newline

\underline{\textbf{Definida por:}}
\newline \url{http://vocab.ciudadesabiertas.es/def/accidente/accid-bici}
\newline

\underline{\textbf{Dominio:}}
\newline LugarAccidente
\newline

\underline{\textbf{Rango:}}
\newline Calle
\newline


\end{flushleft}
\end{mybox}
%----------------------------------------------------------------------------------------------------------------------------------------------------------------------------------------------------------------------------------------------




\begin{mybox}{tieneTipoVehiculo}
\begin{flushleft}
\underline{\textbf{IRI:}}
\url{http://vocab.ciudadesabiertas.es/def/accidente/accid-bici#tieneTipoVehiculo}
\newline

Tipo de vehiculo afectado, p.ej. Bicicleta, Bicicleta EPAC (pedaleo asistido). Se han definido los siguientes elementos:
\newline \url{http://vocab.ciudadesabiertas.es/recurso/accidente/accid-bici/tipoVehiculo/BICICLETA}
\newline \url{http://vocab.ciudadesabiertas.es/recurso/accidente/accid-bici/tipoVehiculo/BICICLETA-EPAC}
\newline

\underline{\textbf{Definida por:}}
\url{http://vocab.ciudadesabiertas.es/def/accidente/accid-bici}
\newline

\underline{\textbf{Dominio:}}
\newline Accidente
\newline

\underline{\textbf{Rango:}}
\newline concept
\newline

\end{flushleft}
\end{mybox}
%----------------------------------------------------------------------------------------------------------------------------------------------------------------------------------------------------------------------------------------------



\begin{mybox}{tieneMeteorologia}
\begin{flushleft}
\underline{\textbf{IRI:}}
\url{http://vocab.ciudadesabiertas.es/def/accidente/accid-bici#tieneMeteorologia}
\newline

Condiciones ambientales que se dan en el momento del siniestro. Se han definido varios tipos posibles:
\newline \url{http://vocab.ciudadesabiertas.es/recurso/accidente/accid-bici/meteorologia/DEPEJADO}
\newline \url{http://vocab.ciudadesabiertas.es/recurso/accidente/accid-bici/meteorologia/LLUVIA-DEBIL}
\newline \url{http://vocab.ciudadesabiertas.es/recurso/accidente/accid-bici/meteorologia/LLUVIA-INTENSA}
\newline \url{http://vocab.ciudadesabiertas.es/recurso/accidente/accid-bici/meteorologia/NUBLADO}
\newline \url{http://vocab.ciudadesabiertas.es/recurso/accidente/accid-bici/meteorologia/GRANIZANDO}
\newline \url{http://vocab.ciudadesabiertas.es/recurso/accidente/accid-bici/meteorologia/DESCONOCIDO}
\newline

\underline{\textbf{Definida por:}}
\url{http://vocab.ciudadesabiertas.es/def/accidente/accid-bici}
\newline

\underline{\textbf{Dominio:}} 
\newline Accidente
\newline

\underline{\textbf{Rango:}}
\newline concept
\newline

\end{flushleft}
\end{mybox}
%----------------------------------------------------------------------------------------------------------------------------------------------------------------------------------------------------------------------------------------------




\begin{mybox}{tieneNumExpediente}
\begin{flushleft}
\underline{\textbf{IRI:}}
\url{http://vocab.ciudadesabiertas.es/def/accidente/accid-bici#tieneNumExpediente}
\newline

El identificador del siniestro que proporciona el ayuntamiento.
\newline

\underline{\textbf{Definida por:}}
\newline \url{http://vocab.ciudadesabiertas.es/def/accidente/accid-bici}
\newline

\underline{\textbf{Dominio:}} 
\newline Accidente
\newline

\underline{\textbf{Rango:}} 
\newline numeroExpediente
\newline

\end{flushleft}
\end{mybox}
%----------------------------------------------------------------------------------------------------------------------------------------------------------------------------------------------------------------------------------------------



\begin{mybox}{tieneTipoAcc}
\begin{flushleft}
\underline{\textbf{IRI:}}
\url{http://vocab.ciudadesabiertas.es/def/accidente/accid-bici#tieneTipoAcc}
\newline

Tipo de accidente asociado. Se han definido para ello varios tipos posibles:
\newline \url{http://vocab.ciudadesabiertas.es/recurso/accidente/accid-bici/tipoAcc/COLISION}
\newline \url{http://vocab.ciudadesabiertas.es/recurso/accidente/accid-bici/tipoAcc/COLISION-DOBLE}
\newline \url{http://vocab.ciudadesabiertas.es/recurso/accidente/accid-bici/tipoAcc/COLISION-MULTIPLE}
\newline \url{http://vocab.ciudadesabiertas.es/recurso/accidente/accid-bici/tipoAcc/ALCANCE}
\newline \url{http://vocab.ciudadesabiertas.es/recurso/accidente/accid-bici/tipoAcc/CHOQUE-NO-VEHICULO}
\newline \url{http://vocab.ciudadesabiertas.es/recurso/accidente/accid-bici/tipoAcc/ATROPELLO-PEATON}
\newline \url{http://vocab.ciudadesabiertas.es/recurso/accidente/accid-bici/tipoAcc/VUELCO}
\newline \url{http://vocab.ciudadesabiertas.es/recurso/accidente/accid-bici/tipoAcc/CAIDA}
\newline \url{http://vocab.ciudadesabiertas.es/recurso/accidente/accid-bici/tipoAcc/OTROS}
\newline

\underline{\textbf{Definida por:}}
\newline \url{http://vocab.ciudadesabiertas.es/def/accidente/accid-bici}
\newline

\underline{\textbf{Dominio:}} 
\newline Accidente
\newline

\underline{\textbf{Rango:}} 
\newline concept
\newline

\end{flushleft}
\end{mybox}
%----------------------------------------------------------------------------------------------------------------------------------------------------------------------------------------------------------------------------------------------




\begin{mybox}{tieneLesividad}
\begin{flushleft}
\underline{\textbf{IRI:}}
\url{http://vocab.ciudadesabiertas.es/def/accidente/accid-bici#tieneLesividad}
\newline

Codigo que indica la gravedad del siniestro para la persona afectada.
\newline

Para su uso se han definido los siguientes elementos:
\newline 01 Atencion en urgencias sin posterior ingreso. - LEVE:
\newline \url{http://vocab.ciudadesabiertas.es/recurso/accidente/accid-bici/lesividad/01}
\newline 02 Ingreso inferior o igual a 24 horas - LEVE:
\newline \url{http://vocab.ciudadesabiertas.es/recurso/accidente/accid-bici/lesividad/02}
\newline 03 Ingreso superior a 24 horas. - GRAVE:
\newline \url{http://vocab.ciudadesabiertas.es/recurso/accidente/accid-bici/lesividad/03}
\newline 04 Fallecido 24 horas - FALLECIDO:
\newline \url{http://vocab.ciudadesabiertas.es/recurso/accidente/accid-bici/lesividad/04}
\newline 05 Asistencia sanitaria ambulatoria con posterioridad - LEVE:
\newline \url{http://vocab.ciudadesabiertas.es/recurso/accidente/accid-bici/lesividad/05}
\newline 06 Asistencia sanitaria inmediata en centro de salud o mutua - LEVE:
\newline \url{http://vocab.ciudadesabiertas.es/recurso/accidente/accid-bici/lesividad/06}
\newline 07 Asistencia sanitaria solo en el lugar del accidente - LEVE:
\newline \url{http://vocab.ciudadesabiertas.es/recurso/accidente/accid-bici/lesividad/07}
\newline 14 Sin asistencia sanitaria:
\newline \url{http://vocab.ciudadesabiertas.es/recurso/accidente/accid-bici/lesividad/14}
\newline 77 Se desconoce:
\newline \url{http://vocab.ciudadesabiertas.es/recurso/accidente/accid-bici/lesividad/77}
\newline


\underline{\textbf{Definida por:}}
\newline \url{http://vocab.ciudadesabiertas.es/def/accidente/accid-bici}
\newline

\underline{\textbf{Dominio:}} 
\newline Accidente
\newline

\underline{\textbf{Rango:}}
\newline concept
\newline

\end{flushleft}
\end{mybox}
%----------------------------------------------------------------------------------------------------------------------------------------------------------------------------------------------------------------------------------------------






\begin{mybox}{gender}
\begin{flushleft}
\underline{\textbf{IRI:}}
\url{https://schema.org/gender}
\newline

Género de la persona afectada.
\newline Seguirá el formato definido por Schema.org \cite{schema_gender}
Se utilizarán las siguientes definidas en la clase:
\newline \url{http://schema.org/Male}
\newline \url{http://schema.org/Female}
\newline \url{http://schema.org/Mixed}
\newline

\underline{\textbf{Definida por:}}
\url{https://schema.org/gender}
\newline

\underline{\textbf{Dominio:}}
\newline PersonaAfectada
\newline

\underline{\textbf{Rango:}}
\newline genderType \cite{schema_gender_explicacion_rango}

\end{flushleft}
\end{mybox}
%----------------------------------------------------------------------------------------------------------------------------------------------------------------------------------------------------------------------------------------------







\begin{mybox}{tieneTipoPersAfect}
\begin{flushleft}
\underline{\textbf{IRI:}}
\url{http://vocab.ciudadesabiertas.es/def/accidente/accid-bici#tieneTipoPersAfect}
\newline

Persona a la que afecta el accidente. Puede ser Conductor, peaton, testigo o viajero. Se han definido los siguientes elementos:
\newline \url{http://vocab.ciudadesabiertas.es/recurso/accidente/accid-bici/persAfec/CONDUCTOR}
\newline \url{http://vocab.ciudadesabiertas.es/recurso/accidente/accid-bici/persAfec/PEATON}
\newline \url{http://vocab.ciudadesabiertas.es/recurso/accidente/accid-bici/persAfec/TESTIGO}
\newline \url{http://vocab.ciudadesabiertas.es/recurso/accidente/accid-bici/persAfec/VIAJERO}
\newline

\underline{\textbf{Definida por:}}
\url{http://vocab.ciudadesabiertas.es/def/accidente/accid-bici}
\newline

\underline{\textbf{Dominio:}}
\newline PersonaAfectada
\newline

\underline{\textbf{Rango:}}
\newline concept
\newline

\end{flushleft}
\end{mybox}
%----------------------------------------------------------------------------------------------------------------------------------------------------------------------------------------------------------------------------------------------






\begin{mybox}{tipoVia}
\begin{flushleft}
\underline{\textbf{IRI:}}
\url{http://vocab.linkeddata.es/datosabiertos/def/urbanismo-infraestructuras/callejero#tipoVia}
\newline

Se ha reutilizado la definición de tipoVia proporcionada por vocab.linkeddata.es \cite{datoabiertos_tipoVia}.
Tipo de vía, que será representado mediante la clasificación en SKOS de URI \url{http://vocab.linkeddata.es/datosabiertos/kos/urbanismo-infraestructuras/tipo-via}. Por ejemplo, estas serán las URIs correspondientes a calles y plazas \url{http://vocab.linkeddata.es/datosabiertos/kos/urbanismo-infraestructuras/tipo-via/CL} \url{http://vocab.linkeddata.es/datosabiertos/kos/urbanismo-infraestructuras/tipo-via/PL}
\newline

\underline{\textbf{Definida por:}}
\url{http://vocab.linkeddata.es/datosabiertos/def/urbanismo-infraestructuras/callejero}
\newline

\underline{\textbf{Dominio:}}
		Calle
\newline

\underline{\textbf{Rango:}}
		concept

\end{flushleft}
\end{mybox}
%----------------------------------------------------------------------------------------------------------------------------------------------------------------------------------------------------------------------------------------------






\begin{mybox}{tieneMunicipio}
\begin{flushleft}
\underline{\textbf{IRI:}}
\url{http://vocab.ciudadesabiertas.es/def/accidente/accid-bici#tieneMunicipio}
\newline

Municipio en el que ocurrió un accidente.
\newline

\underline{\textbf{Definida por:}}
\url{http://vocab.ciudadesabiertas.es/def/accidente/accid-bici}
\newline

\underline{\textbf{Dominio:}}
		LugarAccidente
\newline

\underline{\textbf{Rango:}}
		municipio

\end{flushleft}
\end{mybox}



%----------------------------------------------------------------------------------------------------------------------------------------------------------------------------------------------------------------------------------------------





\begin{mybox}{tieneIdVia}
\begin{flushleft}
\underline{\textbf{IRI:}}
\url{http://vocab.ciudadesabiertas.es/def/accidente/accid-bici#tieneIdVia}
\newline

Identificador de calle asociado.
\newline

\underline{\textbf{Definida por:}}
\url{http://vocab.ciudadesabiertas.es/def/accidente/accid-bici}
\newline

\underline{\textbf{Dominio:}}
		Calle
\newline

\underline{\textbf{Rango:}}
		Via
\newline


\end{flushleft}
\end{mybox}
%----------------------------------------------------------------------------------------------------------------------------------------------------------------------------------------------------------------------------------------------






\begin{mybox}{tienePortal}
\begin{flushleft}
\underline{\textbf{IRI:}}
\url{http://vocab.ciudadesabiertas.es/def/accidente/accid-bici#tienePortal}
\newline

Numero de la calle donde ha ocurrido el accidente, si procede.
\newline

\underline{\textbf{Definida por:}}
\url{http://vocab.ciudadesabiertas.es/def/accidente/accid-bici}
\newline

\underline{\textbf{Dominio:}} 
	Calle
\newline

\underline{\textbf{Rango:}} 
	Portal
\newline

\end{flushleft}
\end{mybox}
%----------------------------------------------------------------------------------------------------------------------------------------------------------------------------------------------------------------------------------------------








\begin{mybox}{esCruce}
\begin{flushleft}
\underline{\textbf{IRI:}}
\url{http://vocab.ciudadesabiertas.es/def/accidente/accid-bici#esCruce}
\newline

Si el accidente ocurrió en un cruce entre 2 o más vías. Se han definido los siguientes elementos:
\newline \url{http://vocab.ciudadesabiertas.es/def/accidente/accid-bici/esCruce/TRUE}
\newline \url{http://vocab.ciudadesabiertas.es/def/accidente/accid-bici/esCruce/FALSE}
%1: Si es entre 2 o mas vias
% - 0: Si es en una unica via
\newline

\underline{\textbf{Definida por:}}
\url{http://vocab.ciudadesabiertas.es/def/accidente/accid-bici}
\newline

\underline{\textbf{Dominio:}} 
	LugarAccidente
\newline

\underline{\textbf{Rango:}}
	concept

\end{flushleft}
\end{mybox}
%----------------------------------------------------------------------------------------------------------------------------------------------------------------------------------------------------------------------------------------------

































